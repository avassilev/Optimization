% !TeX spellcheck = en_GB
\documentclass[10pt]{beamer}
\usetheme{CambridgeUS}
%\usetheme{Boadilla}
\definecolor{myred}{RGB}{163,0,0}
%\usecolortheme[named=blue]{structure}
\usecolortheme{dove}
\usefonttheme[]{professionalfonts}
\usepackage[english]{babel}
\usepackage{amsmath,amsfonts,amssymb}
\usepackage{xcolor}
\usepackage{bm}
\usepackage{gensymb}
\usepackage{verbatim} 
\usepackage{paratype}
\usepackage{mathpazo}
\usepackage{listings}
\lstset{language=Python}

% Number theorem environments
\setbeamertemplate{theorem}[ams style]
\setbeamertemplate{theorems}[numbered]

% Reset theorem-like environments so that each is numbered separately
\usepackage{etoolbox}
\undef{\definition}
\theoremstyle{definition}
\newtheorem{definition}{\translate{Definition}}

% Change colours for theorem-like environments
\definecolor{mygreen1}{RGB}{0,96,0}
\definecolor{mygreen2}{RGB}{229,239,229}
\setbeamercolor{block title}{fg=white,bg=mygreen1}
\setbeamercolor{block body}{fg=black,bg=mygreen2}



\alt<presentation>
{\lstset{%
  basicstyle=\footnotesize\ttfamily,
  commentstyle=\slshape\color{green!50!black},
  frame = single,  
  keywordstyle=\bfseries\color{blue!50!black},
  identifierstyle=\color{blue},
  stringstyle=\color{orange},
  %escapechar=\#,
  showstringspaces = false,
  showtabs = false,
  tabsize = 2,
  emphstyle=\color{red}}
}
{
  \lstset{%
    basicstyle=\ttfamily,
    keywordstyle=\bfseries,
    commentstyle=\itshape,
    escapechar=\#,
    showtabs = false,
	tabsize = 2,
    emphstyle=\bfseries\color{red}
  }
} 

\title{R401: Statistical and Mathematical Foundations}
\subtitle{Lecture 14: Unconstrained Optimization. Static Optimization with Equality Constraints. Lagrange Multipliers}
\author{Andrey Vassilev}

\date{2016/2017} 

\begin{document}
\maketitle

\begin{frame}[fragile]
\frametitle{General Principles and Caveats for the Optimization Module}
\begin{itemize}
\item Emphasis on practicality over rigour
\item Consequently, algorithmic approach and ``recipes'' rather than proofs
\item Also, existence and relevant properties of various objects are often implicitly assumed
\item Pathologies and mathematical peculiarities discussed only in special cases
\end{itemize}
\end{frame}

\begin{frame}[fragile]
\frametitle{Lecture Contents}
\tableofcontents
\end{frame}




\section{Warm-up: Basic Optimization in $ \mathbb{R}^1 $}\label{sec:R1}

\begin{frame}[fragile]
\frametitle{Warm-up: Basic Optimization in $ \mathbb{R}^1 $}

\begin{block}{Fact}
For a differentiable function $ f: \mathbb{R} \rightarrow \mathbb{R} $, a necessary condition for a local extremum is \[ f'(x) = 0. \]
\end{block}
 \bigskip

\begin{exampleblock}{Example}
If $ f(x) = ax^2 + bx + c $, then $ f'(x) = 2ax + b $ and the condition $ f'(x)=0 $ yields the familiar $ x=-\frac{b}{2a} $ (recall your high-school days). Depending on the sign of $ a $, this is a maximum or a minimum (What is the relationship?).
\end{exampleblock}\bigskip

\begin{exampleblock}{Example}
If $ f(x) = x^3 $, then $ f'(x)=3x^2 $ and $ f'(x)=0 \Rightarrow x=0$.

Does the function attain a maximum or a minimum at $ x=0 $?
\end{exampleblock}
\end{frame}

\begin{frame}[fragile]
\frametitle{Warm-up: Basic Optimization in $ \mathbb{R}^1 $}
\begin{figure}
\centering
\includegraphics[width=0.9\linewidth]{cubicfun}
\label{fig:cubicfun}
\end{figure}
\end{frame}

\begin{frame}[fragile]
\frametitle{Warm-up: Basic Optimization in $ \mathbb{R}^1 $}
The answer is ``neither''! The point $ x=0 $ is not a local extremum of $ f(x)=x^3 $.

This illustrates the pitfalls of using necessary conditions to find extrema -- they supply only candidates that need to be checked further.

\begin{alertblock}{Recipe}
\begin{enumerate}
\item 
\item 
\item 
\end{enumerate}
\end{alertblock}
\end{frame}



\end{document}