% !TeX spellcheck = en_GB
\documentclass[10pt]{beamer}
\usetheme{CambridgeUS}
%\usetheme{Boadilla}
\definecolor{myred}{RGB}{163,0,0}
%\usecolortheme[named=blue]{structure}
\usecolortheme{dove}
\usefonttheme[]{professionalfonts}
\usepackage[english]{babel}
\usepackage{amsmath,amsfonts,amssymb}
\usepackage{xcolor}
\usepackage{bm}
\usepackage{gensymb}
\usepackage{verbatim} 
\usepackage{paratype}
\usepackage{mathpazo}
\usepackage{listings}
\lstset{language=Python}

\usepackage{tikz}
\usetikzlibrary{matrix}

% Number theorem environments
\setbeamertemplate{theorem}[ams style]
\setbeamertemplate{theorems}[numbered]

% Reset theorem-like environments so that each is numbered separately
\usepackage{etoolbox}
\undef{\definition}
\theoremstyle{definition}
\newtheorem{definition}{\translate{Definition}}
\newtheorem{Fact}{\translate{Fact}}

% Change colours for theorem-like environments
\definecolor{mygreen1}{RGB}{0,96,0}
\definecolor{mygreen2}{RGB}{229,239,229}
\setbeamercolor{block title}{fg=white,bg=mygreen1}
\setbeamercolor{block body}{fg=black,bg=mygreen2}



\alt<presentation>
{\lstset{%
  basicstyle=\footnotesize\ttfamily,
  commentstyle=\slshape\color{green!50!black},
  frame = single,  
  keywordstyle=\bfseries\color{blue!50!black},
  identifierstyle=\color{blue},
  stringstyle=\color{orange},
  %escapechar=\#,
  showstringspaces = false,
  showtabs = false,
  tabsize = 2,
  emphstyle=\color{red}}
}
{
  \lstset{%
    basicstyle=\ttfamily,
    keywordstyle=\bfseries,
    commentstyle=\itshape,
    escapechar=\#,
    showtabs = false,
	tabsize = 2,
    emphstyle=\bfseries\color{red}
  }
} 

\title{R401: Statistical and Mathematical Foundations}
\subtitle{Lecture 15: Convex Optimization}
\author{Andrey Vassilev}

\date{2016/2017} 

\begin{document}
\maketitle



\begin{frame}[fragile]
\frametitle{Lecture Contents}
\tableofcontents
\end{frame}

\section{Static optimization with inequality constraints}\label{sec:ineq}

\begin{frame}[fragile]
\frametitle{Basic formulation with inequality constraints}
We now look at a problem which is very similar to the case of optimization with equality constraints:

\begin{equation}
f(x_1,\ldots,x_n)\rightarrow \max 
\label{eq:obj}
\end{equation}
s.t.
\begin{equation}
\begin{array}{l l l}
g_1(x_1,\ldots,x_n) \leq b_1\\
g_2(x_1,\ldots,x_n) \leq b_2\\
\cdots \\
g_m(x_1,\ldots,x_n) \leq b_m\\
\end{array}.
\label{eq:constr}
\end{equation}
In vector notation:
\[ f(\mathbf{x}) \rightarrow \max \]
s.t. \[ \mathbf{g}(\mathbf{x})\leq \mathbf{b}. \]
\end{frame}

\begin{frame}[fragile]
\frametitle{Basic formulation with inequality constraints}
\begin{itemize}
\item A vector $ \mathbf{x} $ satisfying the constraints \eqref{eq:constr} is called \emph{admissible} or \emph{feasible}.\bigskip
\item In some alternative (but essentially equivalent) formulations the constraints take the form
$ g_i(x_1,\ldots,x_n)\leq 0 $ or $ g_i(x_1,\ldots,x_n)\geq 0 $ for $ i=1,\ldots,m $.\bigskip
\item The set of admissible vectors is called the \emph{admissible (feasible) set}.\bigskip
\item Note that with inequality constraints the requirement $ m<n $ is not necessary. Intuitively, this is because an inequality constraint is much more forgiving: think of a line vs. a half-plane.\bigskip
\item We focus on maximization problems here. Notice that minimizing a function $ f(x) $ is equivalent to maximizing $ -f(x) $, so there is no loss of generality in our choice.
\end{itemize}
\end{frame}

\begin{frame}[fragile]
\frametitle{}

\end{frame}

\section*{}
\begin{frame}[fragile]
\frametitle{Readings}
Main references:

Syds\ae{}ter et al. \emph{Further mathematics for economic analysis}. Chapter 3.\bigskip

Additional readings:
\end{frame}

\end{document}