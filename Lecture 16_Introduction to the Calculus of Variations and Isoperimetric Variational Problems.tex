% !TeX spellcheck = en_GB
\documentclass[10pt]{beamer}
\usetheme{CambridgeUS}
%\usetheme{Boadilla}
\definecolor{myred}{RGB}{163,0,0}
%\usecolortheme[named=blue]{structure}
\usecolortheme{dove}
\usefonttheme[]{professionalfonts}
\usepackage[english]{babel}
\usepackage{amsmath,amsfonts,amssymb}
\usepackage{xcolor}
\usepackage{bm}
\usepackage{gensymb}
\usepackage{verbatim} 
\usepackage{paratype}
\usepackage{mathpazo}
\usepackage{listings}
\lstset{language=Python}

\usepackage{tikz}
\usetikzlibrary{matrix}

\DeclareMathOperator*{\interior}{int}

% Number theorem environments
\setbeamertemplate{theorem}[ams style]
\setbeamertemplate{theorems}[numbered]

% Reset theorem-like environments so that each is numbered separately
\usepackage{etoolbox}
\undef{\definition}
\theoremstyle{definition}
\newtheorem{definition}{\translate{Definition}}
\newtheorem{Fact}{\translate{Fact}}

% Change colours for theorem-like environments
\definecolor{mygreen1}{RGB}{0,96,0}
\definecolor{mygreen2}{RGB}{229,239,229}
\setbeamercolor{block title}{fg=white,bg=mygreen1}
\setbeamercolor{block body}{fg=black,bg=mygreen2}



\alt<presentation>
{\lstset{%
  basicstyle=\footnotesize\ttfamily,
  commentstyle=\slshape\color{green!50!black},
  frame = single,  
  keywordstyle=\bfseries\color{blue!50!black},
  identifierstyle=\color{blue},
  stringstyle=\color{orange},
  %escapechar=\#,
  showstringspaces = false,
  showtabs = false,
  tabsize = 2,
  emphstyle=\color{red}}
}
{
  \lstset{%
    basicstyle=\ttfamily,
    keywordstyle=\bfseries,
    commentstyle=\itshape,
    escapechar=\#,
    showtabs = false,
	tabsize = 2,
    emphstyle=\bfseries\color{red}
  }
} 

\title{R401: Statistical and Mathematical Foundations}
\subtitle{Lecture 16: Introduction to the Calculus of Variations. Isoperimetric Variational Problems.}
\author{Andrey Vassilev}

\date{2016/2017} 
    
\AtBeginSection{\frame{\usebeamerfont{section title}\centering\insertsection}}

\begin{document}
\maketitle



\begin{frame}[fragile]
\frametitle{Lecture Contents}
\tableofcontents
\end{frame}

\begin{section}{An abstract look at optimization problems}\label{sec:abstr}
\begin{frame}[fragile]
\frametitle{What is an optimization problem, really?}
\begin{itemize} \itemsep1em
\item You have already seen various optimization problems.
\item They required finding maxima or minima of functions.
\item These functions involved one or several variables.
\item In some cases these variables were constrained by (systems of) equations or inequalities.
\end{itemize}\bigskip \pause

\alert{What are the common features of the optimization problems that we have encountered?}
\end{frame}

\begin{frame}[fragile]
\frametitle{The constituents of an optimization problem}
From the examples of optimization problems we can extract the following:
\begin{itemize} \itemsep1em
\item An optimization problem requires a function to be maximized or minimized (the \emph{objective function}).
\item In the most abstract sense, a function $ f $ is a rule of association (a mapping) between the elements of a set $ X $ and the elements of another set $ Y $:
\[ f: X \rightarrow Y. \] The sets $ X $ (the \emph{domain} of $ f $) and $ Y $ (the \emph{codomain} of $ f $) can be of arbitrary nature.
\item Examples of functions under this broadened definition include:
	\begin{itemize}
	\item The familiar functions like $ f(x) = 2x^4 $ are mappings from (subsets of) $ \mathbb{R}^1 $ to (subsets of) $ \mathbb{R}^1 $. 
	\item Functions of $ n $ real variables are mappings from $ \mathbb{R}^n $ to  $ \mathbb{R}^1 $.
	\item An $ m \times n $ matrix with real entries can serve to define a linear function from $ \mathbb{R}^n $ to $ \mathbb{R}^m $ via post-multiplication with vectors $ \mathbf{x} \in \mathbb{R}^n $.
	\item The determinant maps the set of square matrices to the set of real numbers.
	\item The definite integral can be viewed as a mapping from a suitable set of functions to the (extended) real numbers.
	\end{itemize}
\end{itemize}
\end{frame}

\begin{frame}[fragile]
\frametitle{The constituents of an optimization problem}
\begin{itemize} \itemsep1em
\item The set of functions with domain $ X $ and codomain $ Y $ is denoted by $ Y^X $. 
		\begin{itemize}
		\item Thus, the well-known $ \mathbb{R}^n $ can be reinterpreted as a space of functions from the set $ \{1,2,\ldots,n\} $ to the real numbers.
		\item Similarly, the notation $ \mathbb{R}^{m\times n} $ for the real-valued matrices with $ m $ rows and $ n $ columns tells us that a matrix can be viewed as a function from the Cartesian product of the sets $ \{1,2,\ldots,m\} $ and $ \{1,2,\ldots,n\} $ to $ \mathbb{R} $.
		\end{itemize} 
\item In order for a function to be suitable for use as an objective function, the elements of its codomain must be ordered (that is, we must be able to tell that one is ``larger'' or ``better'' than another).
\item For practical purposes this means that we'll be optimizing functions having $ \mathbb{R}^1 $ as their codomain. 
\item In contrast, the domains of the objective functions can be more diverse types of sets, as long as we have appropriate methods to choose among the different elements.
\end{itemize}
\end{frame}

\begin{frame}[fragile]
\frametitle{The constituents of an optimization problem}
\begin{itemize} \itemsep1em
\item The elements of the domain of an objective function can be subject to various requirements -- the \emph{constraints} of the optimization problem.
\item Typically, these constraints take the form of other functions (with the same domain as the objective function), which are used to define various equations or inequalities.
\item This idea remains intact irrespective of the types of sets that serve as domains of the constraint functions.
\end{itemize}
\end{frame}

\begin{frame}[fragile]
\frametitle{Specific classes of optimization problems}

\end{frame}

\end{section}
\begin{frame}[fragile]
\frametitle{Readings}
Main references:

Elsgolts. \emph{Differential equations and the calculus of variations}. Chapters 6 and 9.\bigskip

Additional readings:

Syds\ae{}ter et al. [SHSS] \emph{Further mathematics for economic analysis}. Chapter 8.

\end{frame}

\end{document}