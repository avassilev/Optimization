% !TeX spellcheck = en_GB
\documentclass[10pt]{beamer}
\usetheme{CambridgeUS}
%\usetheme{Boadilla}
\definecolor{myred}{RGB}{163,0,0}
%\usecolortheme[named=blue]{structure}
\usecolortheme{dove}
\usefonttheme[]{professionalfonts}
\usepackage[english]{babel}
\usepackage{amsmath,amsfonts,amssymb}
\usepackage{xcolor}
\usepackage{bm}
\usepackage{gensymb}
\usepackage{verbatim} 
\usepackage{paratype}
\usepackage{mathpazo}
\usepackage{listings}
\lstset{language=Python}

\usepackage{tikz}
\usetikzlibrary{matrix}

\DeclareMathOperator*{\interior}{int}

% Number theorem environments
\setbeamertemplate{theorem}[ams style]
\setbeamertemplate{theorems}[numbered]

% Reset theorem-like environments so that each is numbered separately
\usepackage{etoolbox}
\undef{\definition}
\theoremstyle{definition}
\newtheorem{definition}{\translate{Definition}}
\newtheorem{Fact}{\translate{Fact}}

% Change colours for theorem-like environments
\definecolor{mygreen1}{RGB}{0,96,0}
\definecolor{mygreen2}{RGB}{229,239,229}
\setbeamercolor{block title}{fg=white,bg=mygreen1}
\setbeamercolor{block body}{fg=black,bg=mygreen2}



\alt<presentation>
{\lstset{%
  basicstyle=\footnotesize\ttfamily,
  commentstyle=\slshape\color{green!50!black},
  frame = single,  
  keywordstyle=\bfseries\color{blue!50!black},
  identifierstyle=\color{blue},
  stringstyle=\color{orange},
  %escapechar=\#,
  showstringspaces = false,
  showtabs = false,
  tabsize = 2,
  emphstyle=\color{red}}
}
{
  \lstset{%
    basicstyle=\ttfamily,
    keywordstyle=\bfseries,
    commentstyle=\itshape,
    escapechar=\#,
    showtabs = false,
	tabsize = 2,
    emphstyle=\bfseries\color{red}
  }
} 

\title{R401: Statistical and Mathematical Foundations \bigskip}
\subtitle{\textcolor{myred}{Deterministic Optimal Control in Continuous Time: The Infinite Horizon Case}}
\author{Andrey Vassilev}

\date{} 
    
\AtBeginSection{\frame{\usebeamerfont{section title}\centering\insertsection}}

\begin{document}
\maketitle



\begin{frame}[fragile]
\frametitle{Lecture Contents}
\tableofcontents
\end{frame}

\begin{section}{Introduction}\label{sec:intro}

\begin{frame}[fragile]
\frametitle{The rationale behind infinite horizons}
\begin{itemize}\itemsep1em
\item We now start studying an important class of optimal control problems for which there is no finite terminal time $ T $. Thus, the objective functional will look like \[ \int_{0}^{\infty} F(x(t),u(t),t)\,dt \] or a version thereof.\pause
\item \alert{Why do we need the infinite planning horizon?}\pause
\item After all, people are mortal and we'll stop planning one day...
\end{itemize}
\end{frame}

\begin{frame}[fragile]
\frametitle{The rationale behind infinite horizons}
\begin{itemize}\itemsep1em
\item There are two (related) economic reasons why an infinite-horizon formulation might be appropriate:
\begin{enumerate}
\item Entities such as households and firms may exist indefinitely despite turnover in their composition (i.e. family members dying or moving, employees changing jobs etc.).
\item Often there is uncertainty about the end of the planning horizon. This can be conveniently modelled as an infinite horizon, especially when it is reasonable to assume that the true, finite horizon is sufficiently distant.
\end{enumerate}
\item A technical complication with finite planning horizons arises when state variables represent economically valuable resources (wealth, capital). In these common cases we need to either:
	\begin{itemize}
	\item exhaust the respective resource fully as required by optimality if there is no scrap value,  which is often implausible, or
	\item specify an appropriate scrap value term in the objective function, which may be difficult.
	\end{itemize}
\end{itemize}
\end{frame}

\begin{frame}[fragile]
\frametitle{The rationale behind infinite horizons}
\begin{itemize}\itemsep1em
\item Apart from matters of interpretation, an infinite-horizon formulation eliminates some mathematical difficulties (one generally obtains simpler and cleaner expressions).
\item However, this is not costless, as certain other complications arise.
\item More specifically, we need to modify appropriately our definition of optimality to capture situations that arise in the case of an infinite horizon.
\end{itemize}
\end{frame}

\begin{frame}[fragile]
\frametitle{The basic problem}
The problem we shall be studying is the following:
\begin{equation}\begin{split}
&\max_{u(t)\in \Omega(t)}~\int_{0}^{\infty}F(x(t),u(t),t)\,dt\\
&\text{s.t.}\\
&\dot{x}(t)=f(x(t),u(t),t),\quad x(0)=x_0.
\end{split}
\label{eq:InfHorOCproblem}
\end{equation}

Sometimes problem \eqref{eq:InfHorOCproblem} is replaced by a simplified version that reflects the structure of typical economic problems:
\begin{equation}\begin{split}
&\max_{u(t)\in \Omega(t)}~\int_{0}^{\infty}e^{-\rho t} \phi(x(t),u(t))\,dt, \quad \rho>0,\\
&\text{s.t.}\\
&\dot{x}(t)=f(x(t),u(t)),\quad x(0)=x_0.
\end{split}
\label{eq:simpleInfHorOCproblem}
\end{equation}
\end{frame}

\begin{frame}[fragile]
\frametitle{Specificities of the infinite-horizon setting}
\begin{itemize}\itemsep1em
\item One obvious requirement in order to have a well-defined problem is for the improper integral forming the objective functional to converge for every admissible state-control pair.
\item This leads to a relatively simple case, obtainable for instance for problem \eqref{eq:simpleInfHorOCproblem} when \[ |\phi(x,u)|\leq M,~\forall (x,u). \]
\item Unfortunately many problems do not admit such a clean characterization, necessitating a generalization of optimality criteria.
\end{itemize}
\end{frame}

\begin{frame}[fragile]
\frametitle{Specificities of the infinite-horizon setting}
\begin{itemize}\itemsep1em
\item It is possible to have situations in which the objective functional does not converge, yet for some state-control pair its value is greater than the value for every alternative state-control pair for every finite time horizon.
\item In such cases it is natural to think of the ``dominant'' state-control pair as optimal, even though the verification is more complex than comparing two numbers.
\item Other possibilities for defining optimality exist as well.
\end{itemize}
\end{frame}

\begin{frame}[fragile]
\frametitle{Optimality criteria in infinite-horizon problems}
Among popular optimality criteria when the integral in \eqref{eq:InfHorOCproblem} or \eqref{eq:simpleInfHorOCproblem} does not converge, the most permissive definition is the following:
\begin{definition}[Piecewise (PW) optimality]
An admissible pair $ (x^*(t),u^*(t)) $ is \emph{piecewise optimal} if for every $ T\geq 0 $, the restriction of $ (x^*(t),u^*(t)) $ to $ [0,T] $ is optimal for the corresponding fixed horizon problem with terminal condition $ x(T)=x^*(T) $ and $ \int_{0}^{T}F(x(t),u(t),t)\,dt $ as the objective functional.
\label{def:PWopt}
\end{definition}
\end{frame}

\begin{frame}[fragile]
\frametitle{Optimality criteria in infinite-horizon problems}
In contrast, one of the stricter definitions is that of overtaking optimality.

We need to define the quantity \[ D(t):= \int_{0}^{t}F(x^*(\tau),u^*(\tau),\tau)\,d\tau - \int_{0}^{t}F(x(\tau),u(\tau),\tau)\,d\tau.\]

\begin{definition}[Overtaking (OT) optimality]
An admissible pair $ (x^*(t),u^*(t)) $ is \emph{overtaking optimal} if there exists a number $ t' $ such that $ D(t)\geq 0,~ \forall t\geq t'. $
\label{def:OTopt}
\end{definition}\bigskip

It can be shown that overtaking optimality implies piecewise optimality.\bigskip

There are definitions of optimality that take an intermediate position between PW optimality and OT optimality in terms of strictness.
\end{frame}

\begin{frame}[fragile]
\frametitle{Pitfalls of applying the transversality condition in infinite-horizon problems}
\begin{itemize}\itemsep1em
\item In the finite-horizon setting the transversality conditions impose constraints on the terminal values of the adjoint variables $ \lambda(T) $. 
\item In the absence of a salvage value term they take the form $ \lambda(T)=0 $.
\item One is tempted to conjecture that the natural extension of this requirement to the infinite-horizon setting would be \[ \lim\limits_{T\rightarrow \infty}\lambda(T)=0 . \]
\item Unfortunately there are counterexamples showing that in general this is not the case and additional modifications are required to obtain NCs or SCs for optimality.
\end{itemize}
\end{frame}
\end{section}

\begin{section}{Optimality conditions in the infinite-horizon setting}\label{sec:NCsSCs}

\begin{frame}[fragile]
\frametitle{Necessary conditions for PW optimality}
The following result by H. Halkin provides NCs at the expense of a slight complication of the formulation of the Hamiltonian.

\begin{Fact}[Halkin]
Let $ (x^*(t),u^*(t) ) $ be a PW-optimal state-control pair for the problem \eqref{eq:InfHorOCproblem} and define the Hamiltonian as \begin{equation}
H = \textcolor{red}{\lambda_0} F(x,u,t) + \sum_{i=1}^{n}\lambda_i f_i(x,u,t).
\label{eq:HamiltP0}
\end{equation}

Then there exist a constant $ \lambda_0 $ and (piecewise) continuously differentiable functions $ \lambda_i(t),~i=1,\ldots,n $, such that for all $ t $
\begin{enumerate}
\item $ (\lambda_0,\lambda(t)) \neq (0,\ldots,0)$,
\item $ H(x^*(t),u^*(t),\lambda(t),t )\geq H(x^*(t),u(t),\lambda(t),t ),~\forall u(t)\in \Omega(t) $,
\item $ \dot{\lambda}(t) = -\nabla_x H(x^*(t),u^*(t),\lambda(t),t) $.
\end{enumerate}

Moreover, either $ \lambda_0=0 $ or $ \lambda_0=1 $.
\label{fc:HalkinNCs}
\end{Fact}
\end{frame}

\begin{frame}[fragile]
\frametitle{A comment on necessary conditions}
\begin{itemize}\itemsep1em
\item Halkin's necessary conditions encompass a broad range of candidates and may create difficulties in filtering potential solutions.
\item There exist other versions of necessary conditions which one may try to apply (see e.g. SS, Sec. 3.9).
\item However, some of them are rather cumbersome to apply.
\item As a result, using appropriate sufficient conditions for a well-structured problem may be your best bet.
\end{itemize}
\end{frame}

\begin{frame}[fragile]
\frametitle{Sufficient conditions for OT optimality}
Fact \ref{fc:HalkinNCs} is relatively easy to check but often provides too many candidates. Using sufficiency results is a partial solution to this. \pause

\begin{Fact}[Mangasarian-type sufficiency]
Consider the problem \eqref{eq:InfHorOCproblem}, assuming that the set $ \Omega $ is convex and $ F,f $ are continuously differentiable with respect to $ u $. Suppose $ (x^*,u^*) $ is an admissible pair and there exists a continuous function $ \lambda(t) $ such that, using the Hamiltonian \eqref{eq:HamiltP0}, for $ \lambda_0=1 $ and $ t\geq 0 $ we have
\begin{enumerate}
\item $ u^*(t) $ maximizes $ H(x^*(t),u(t),\lambda(t),t )$ for $ u(t)\in \Omega $,
\item $ \dot{\lambda}(t) = -\nabla_x H(x^*(t),u^*(t),\lambda(t),t) $,
\item $ H(x,u,\lambda(t),t ) $ is concave in $ (x,u),~\forall t \geq 0 $.
\end{enumerate}

Then the following transversality condition guarantees that $ (x^*,u^*) $ is overtaking optimal:

For all admissible $ x(t) $ there exists a number $ t' $ such that \newline
${} \qquad \qquad \lambda'(t) (x(t)-x^*(t))\geq 0,\quad t\geq t'. $ 
\label{fc:SCsOT}
\end{Fact}
\end{frame}

\begin{frame}[fragile]
\frametitle{Comments on the SCs for OT optimality}
\begin{itemize}\itemsep1em
\item Fact \ref{fc:SCsOT} provides sufficiency conditions for a fairly general problem and a strict definition of optimality.
\item One can resort to a more permissive definition of optimality (e.g. the so-called \emph{catching up optimality}, see SS, Ch. 3, or SHSS, Sec. 10.3) at the expense of additional mathematical complications.
\item The price we pay in any case is that the transversality condition is rather cumbersome to check, as it requires us to compare a candidate trajectory $ x^* $ to all feasible alternatives $ x $.
\item This is realistic only in special cases.
\item As a result, a typical approach in economic applications is to resort to conditions that are somewhat easier to check but are applicable to less general problems.
\end{itemize}
\end{frame}

\begin{frame}[fragile]
\frametitle{A sufficiency result in a less general setting}
\begin{itemize}\itemsep1em
\item Suppose we have a simplified version of problem \eqref{eq:simpleInfHorOCproblem} with only \textbf{one} state variable $ x(t) $ and \textbf{one} control variable $ u(t) $ taking values in a constant region $ \Omega $:\begin{equation}\begin{split}
&\max_{u(t)}~\int_{0}^{\infty}e^{-\rho t} \phi(x(t),u(t))\,dt, \quad \rho>0,\\
&\text{s.t.}\\
&\dot{x}(t)=f(x(t),u(t)),\quad x(0)=x_0,\quad u(t)\in \Omega.
\end{split}
\label{eq:simplerInfHorOCproblem}
\end{equation}
\item Suppose also that the integral in \eqref{eq:simplerInfHorOCproblem} converges for all admissible $ (x,u) $.
\item We therefore can employ the conventional definition of optimality.
\item Also, because of the specific discounting structure, we can work in a current-value setting.
\end{itemize}
\end{frame}

\begin{frame}[fragile]
\frametitle{A sufficiency result in a less general setting}
\begin{itemize}\itemsep1em
\item Construct the current-value Hamiltonian \[ H^c (x,u,\lambda) = \lambda_0 \phi(x,u) + \lambda f(x,u). \]

\begin{Fact}
Let an admissible pair $ (x^*(t),u^*(t)) $ for problem \eqref{eq:simplerInfHorOCproblem} satisfy the following conditions for some $ \lambda(t),~\forall t \geq 0 $, with $ \lambda_0=1 $:
\begin{enumerate}
\item $ u^*(t) $ maximizes $ H^c(x^*, u, \lambda) $ w.r.t. $ u \in \Omega $,
\item $ \dot{\lambda}(t) -\rho \lambda(t) = -\frac{\partial}{\partial x}H^c(x^*(t),u^*(t),\lambda(t)) $,
\item $ H^c(x,u,\lambda(t)) $ is concave w.r.t. $ (x,u) $,
\item $ \lim\limits_{t\rightarrow \infty} \lambda(t)e^{-\rho t}(x(t)-x^*(t)) \geq 0$ for all admissible $ x(t) $.
\end{enumerate}

Then the pair $ (x^*(t),u^*(t)) $ is optimal.
\label{fc:simpleSC}
\end{Fact}
\end{itemize}
\end{frame}

\begin{frame}[fragile]
\frametitle{Further comments}
\begin{itemize}\itemsep1em
\item While the condition $ \lim\limits_{t\rightarrow \infty}\lambda(t)=0 $ is not correct in a strict mathematical sense as an optimality condition, it does turn out to hold in many economic models when there are no constraints on $ x(t) $ for $ t\rightarrow \infty $.
\item Fact \ref{fc:simpleSC} is somewhat easier to check because of the presence of the exponential term $ e^{-\rho t} $ in the transversality condition, yet some care is still needed.
\item One situation in which the transversality condition may be shown to hold is the following:
	\begin{itemize}\itemsep1em
	\item $ x(t)\geq 0,~\forall t$,
	\item $ \lim\limits_{t\rightarrow \infty} \lambda(t)e^{-\rho t} = 0 $,
	\item $ \lim\limits_{t\rightarrow \infty} \lambda(t)e^{-\rho t}x^*(t) = 0 $,
	\item $ \lambda(t)\geq 0 $ for $ t $ large enough.
	\end{itemize}
\end{itemize}
\end{frame}

\begin{frame}[fragile]
\frametitle{Further comments}
\begin{itemize}\itemsep1em
\item An even more specific instance of the previous case would be a situation in which:
	\begin{itemize}\itemsep1em
	\item  $ x(t)\geq 0,~\forall t$ (This is the case when the state variables are naturally nonnegative economic quantities such as capital or gross wealth.),
	\item $ \lim\limits_{t\rightarrow \infty} x^*(t) = \bar{x} \geq 0 $,
	\item $ \lim\limits_{t\rightarrow \infty} \lambda(t) = \bar{\lambda} \geq 0 $.
	\end{itemize}
\item Apart from ensuring that the transversality condition holds, this situation is convenient from an economic standpoint: it can be interpreted as the system tending to a \emph{(stationary) equilibrium}.
\item This points to the need to be able to say something about the behaviour of the differential equations system for $ x(t) $ and $ \lambda(t) $ even in cases when it is not explicitly solvable.
\end{itemize}
\end{frame}
\end{section}

\begin{section}{Phase-space analysis}\label{sec:phspace}

\begin{frame}[fragile]
\frametitle{General facts about phase-space analysis}
\begin{itemize}\itemsep1em
\item Phase-space analysis is a method for studying a system of differential equations in qualitative terms. 
\item Roughly, this means we establish where the system is going under certain conditions rather than look for explicit solutions.
\item In the context of optimal control, we can use it to improve our understanding of how the system of state and adjoint variables behaves.
\item Phase-space analysis works well in two dimensions (the geometry there is obvious) but becomes unwieldy for problems of higher dimensionality.
\item It is also readily applicable to systems that do not depend explicitly on time, i.e. autonomous systems.
\item Consequently, we can tackle Problem \eqref{eq:simplerInfHorOCproblem} by means of such methods.
\end{itemize}
\end{frame}

\begin{frame}[fragile]
\frametitle{The setup}
\begin{itemize}\itemsep1em
\item Notice that a solution $ u^* $ to the Hamiltonian maximization problem 
\[ \max_{u\in \Omega} \left \{\phi (x,u)+\lambda f(x,u)\right \} \] is a function of $ x $ and $ \lambda $, i.e. $ u^*=u(x,\lambda) $.
\item This means that \[ \dot{x}=f(x,u^*)\textcolor{red}{=F(x,\lambda)} \] and \[\dot{\lambda} = \rho \lambda - \frac{\partial}{\partial x}H^c (x,u^*,\lambda) \textcolor{red}{= G(x,\lambda)}. \]
\item We shall study the case where $ F $ and $ G $ are $ C^1 $ functions.
\end{itemize}
\end{frame} 

\begin{frame}[fragile]
\frametitle{The setup}
\begin{itemize}\itemsep1em
\item The system \begin{equation}\label{eq:phsystem}
\begin{split}
& \dot{x}=F(x,\lambda),\\
& \dot{\lambda} = G(x,\lambda)
\end{split} \end{equation} defines a family of curves in $ x\lambda $ space, the \emph{phase space} of the system.
\item For any point in $ x\lambda $ space, the system \eqref{eq:phsystem} defines a velocity vector $ (\dot{x},\dot{\lambda}) $.
\item The velocity vectors form the \emph{vector field} of the system.
\end{itemize}
\end{frame}

\begin{frame}[fragile]
\frametitle{A first look at the phase space}
\begin{columns}
\column{.4\textwidth}
\begin{itemize}\itemsep1em
\item One can represent the properties of the state space through the components of the velocity vectors (the black pairs of arrows)...
\item ...or directly through the resulting vectors (\textcolor{red}{the red arrows}).
\item They both show parts of the vector field, indicating the direction of movement that will take place if we are at a particular point $ (x,\lambda) $.
\end{itemize}
\column{.4\textwidth}
\begin{center}
% Graphic for TeX using PGF
% Title: S:\ANDREY\FEBA\Applied_Econometrics_Econ_Modelling_MSc\Optimization\SS1.dia
% Creator: Dia v0.97.2
% CreationDate: Fri Nov 18 23:12:05 2016
% For: User
% \usepackage{tikz}
% The following commands are not supported in PSTricks at present
% We define them conditionally, so when they are implemented,
% this pgf file will use them.
\ifx\du\undefined
  \newlength{\du}
\fi
\setlength{\du}{15\unitlength}
\begin{tikzpicture}[scale = 0.6]
\pgftransformxscale{1.000000}
\pgftransformyscale{-1.000000}
\definecolor{dialinecolor}{rgb}{0.000000, 0.000000, 0.000000}
\pgfsetstrokecolor{dialinecolor}
\definecolor{dialinecolor}{rgb}{1.000000, 1.000000, 1.000000}
\pgfsetfillcolor{dialinecolor}
\pgfsetlinewidth{0.030000\du}
\pgfsetdash{}{0pt}
\pgfsetdash{}{0pt}
\pgfsetbuttcap
{
\definecolor{dialinecolor}{rgb}{0.000000, 0.000000, 0.000000}
\pgfsetfillcolor{dialinecolor}
% was here!!!
\pgfsetarrowsend{stealth}
\definecolor{dialinecolor}{rgb}{0.000000, 0.000000, 0.000000}
\pgfsetstrokecolor{dialinecolor}
\draw (5.000000\du,5.000000\du)--(20.027400\du,4.978920\du);
}
\pgfsetlinewidth{0.030000\du}
\pgfsetdash{}{0pt}
\pgfsetdash{}{0pt}
\pgfsetbuttcap
{
\definecolor{dialinecolor}{rgb}{0.000000, 0.000000, 0.000000}
\pgfsetfillcolor{dialinecolor}
% was here!!!
\pgfsetarrowsend{stealth}
\definecolor{dialinecolor}{rgb}{0.000000, 0.000000, 0.000000}
\pgfsetstrokecolor{dialinecolor}
\draw (9.000000\du,7.600000\du)--(9.000000\du,-4.000000\du);
}
% setfont left to latex
\definecolor{dialinecolor}{rgb}{0.000000, 0.000000, 0.000000}
\pgfsetstrokecolor{dialinecolor}
\node[anchor=west] at (7.681980\du,-3.5\du){$\lambda$};
% setfont left to latex
\definecolor{dialinecolor}{rgb}{0.000000, 0.000000, 0.000000}
\pgfsetstrokecolor{dialinecolor}
\node[anchor=west] at (19.918800\du,5.645820\du){$x$};
\pgfsetlinewidth{0.030000\du}
\pgfsetdash{}{0pt}
\pgfsetdash{}{0pt}
\pgfsetbuttcap
{
\definecolor{dialinecolor}{rgb}{0.000000, 0.000000, 0.000000}
\pgfsetfillcolor{dialinecolor}
% was here!!!
\pgfsetarrowsend{stealth}
\definecolor{dialinecolor}{rgb}{0.000000, 0.000000, 0.000000}
\pgfsetstrokecolor{dialinecolor}
\draw (10.200000\du,2.000000\du)--(10.200000\du,1.000000\du);
}
\pgfsetlinewidth{0.030000\du}
\pgfsetdash{}{0pt}
\pgfsetdash{}{0pt}
\pgfsetbuttcap
{
\definecolor{dialinecolor}{rgb}{0.000000, 0.000000, 0.000000}
\pgfsetfillcolor{dialinecolor}
% was here!!!
\pgfsetarrowsend{stealth}
\definecolor{dialinecolor}{rgb}{0.000000, 0.000000, 0.000000}
\pgfsetstrokecolor{dialinecolor}
\draw (10.185600\du,2.000000\du)--(11.185600\du,2.000000\du);
}
\pgfsetlinewidth{0.030000\du}
\pgfsetdash{}{0pt}
\pgfsetdash{}{0pt}
\pgfsetbuttcap
{
\definecolor{dialinecolor}{rgb}{0.000000, 0.000000, 0.000000}
\pgfsetfillcolor{dialinecolor}
% was here!!!
\pgfsetarrowsend{stealth}
\definecolor{dialinecolor}{rgb}{0.000000, 0.000000, 0.000000}
\pgfsetstrokecolor{dialinecolor}
\draw[red] (10.199000\du,1.995060\du)--(11.176100\du,1.003420\du);
}
\pgfsetlinewidth{0.030000\du}
\pgfsetdash{}{0pt}
\pgfsetdash{}{0pt}
\pgfsetbuttcap
{
\definecolor{dialinecolor}{rgb}{0.000000, 0.000000, 0.000000}
\pgfsetfillcolor{dialinecolor}
% was here!!!
\pgfsetarrowsend{stealth}
\definecolor{dialinecolor}{rgb}{0.000000, 0.000000, 0.000000}
\pgfsetstrokecolor{dialinecolor}
\draw (7.000000\du,3.000000\du)--(7.000000\du,4.400000\du);
}
\pgfsetlinewidth{0.030000\du}
\pgfsetdash{}{0pt}
\pgfsetdash{}{0pt}
\pgfsetbuttcap
{
\definecolor{dialinecolor}{rgb}{0.000000, 0.000000, 0.000000}
\pgfsetfillcolor{dialinecolor}
% was here!!!
\pgfsetarrowsend{stealth}
\definecolor{dialinecolor}{rgb}{0.000000, 0.000000, 0.000000}
\pgfsetstrokecolor{dialinecolor}
\draw (7.014680\du,3.000000\du)--(5.994680\du,3.000000\du);
}
\pgfsetlinewidth{0.030000\du}
\pgfsetdash{}{0pt}
\pgfsetdash{}{0pt}
\pgfsetbuttcap
{
\definecolor{dialinecolor}{rgb}{0.000000, 0.000000, 0.000000}
\pgfsetfillcolor{dialinecolor}
% was here!!!
\pgfsetarrowsend{stealth}
\definecolor{dialinecolor}{rgb}{0.000000, 0.000000, 0.000000}
\pgfsetstrokecolor{dialinecolor}
\draw[red] (7.000000\du,3.000000\du)--(6.000000\du,4.400000\du);
}
\end{tikzpicture}

\end{center}
\end{columns}
\end{frame}

\begin{frame}[fragile]
\frametitle{A first look at the phase space}
\begin{columns}
\column{.4\textwidth}
\begin{itemize}\itemsep1em
\item Sometimes instead of the vector field we draw examples of \emph{solution paths} (\emph{trajectories})
\item SS call these \emph{integral curves} although the literature seems to differ on the use of the term.
\item They are often drawn with arrows indicating the direction in which a particular curve is traversed over time.
\end{itemize}
\column{.4\textwidth}
\begin{center}
% Graphic for TeX using PGF
% Title: S:\ANDREY\FEBA\Applied_Econometrics_Econ_Modelling_MSc\Optimization\SS2.dia
% Creator: Dia v0.97.2
% CreationDate: Sat Nov 19 09:07:30 2016
% For: User
% \usepackage{tikz}
% The following commands are not supported in PSTricks at present
% We define them conditionally, so when they are implemented,
% this pgf file will use them.
\ifx\du\undefined
  \newlength{\du}
\fi
\setlength{\du}{15\unitlength}
\begin{tikzpicture}[scale = 0.6]
\pgftransformxscale{1.000000}
\pgftransformyscale{-1.000000}
\definecolor{dialinecolor}{rgb}{0.000000, 0.000000, 0.000000}
\pgfsetstrokecolor{dialinecolor}
\definecolor{dialinecolor}{rgb}{1.000000, 1.000000, 1.000000}
\pgfsetfillcolor{dialinecolor}
\pgfsetlinewidth{0.030000\du}
\pgfsetdash{}{0pt}
\pgfsetdash{}{0pt}
\pgfsetbuttcap
{
\definecolor{dialinecolor}{rgb}{0.000000, 0.000000, 0.000000}
\pgfsetfillcolor{dialinecolor}
% was here!!!
\pgfsetarrowsend{stealth}
\definecolor{dialinecolor}{rgb}{0.000000, 0.000000, 0.000000}
\pgfsetstrokecolor{dialinecolor}
\draw (5.000000\du,5.000000\du)--(20.027400\du,4.978920\du);
}
\pgfsetlinewidth{0.030000\du}
\pgfsetdash{}{0pt}
\pgfsetdash{}{0pt}
\pgfsetbuttcap
{
\definecolor{dialinecolor}{rgb}{0.000000, 0.000000, 0.000000}
\pgfsetfillcolor{dialinecolor}
% was here!!!
\pgfsetarrowsend{stealth}
\definecolor{dialinecolor}{rgb}{0.000000, 0.000000, 0.000000}
\pgfsetstrokecolor{dialinecolor}
\draw (9.000000\du,7.600000\du)--(9.000000\du,-4.000000\du);
}
% setfont left to latex
\definecolor{dialinecolor}{rgb}{0.000000, 0.000000, 0.000000}
\pgfsetstrokecolor{dialinecolor}
\node[anchor=west] at (7.681980\du,-3.287520\du){$ \lambda $};
% setfont left to latex
\definecolor{dialinecolor}{rgb}{0.000000, 0.000000, 0.000000}
\pgfsetstrokecolor{dialinecolor}
\node[anchor=west] at (19.918800\du,5.645820\du){$ x $};
\pgfsetlinewidth{0.050000\du}
\pgfsetdash{}{0pt}
\pgfsetdash{}{0pt}
\pgfsetmiterjoin
\pgfsetbuttcap
{
\definecolor{dialinecolor}{rgb}{0.000000, 0.000000, 0.000000}
\pgfsetfillcolor{dialinecolor}
% was here!!!
\definecolor{dialinecolor}{rgb}{0.000000, 0.000000, 0.000000}
\pgfsetstrokecolor{dialinecolor}
\pgfpathmoveto{\pgfpoint{5.278557\du}{-2.212002\du}}
\pgfpathcurveto{\pgfpoint{6.845155\du}{0.619374\du}}{\pgfpoint{6.945763\du}{3.493867\du}}{\pgfpoint{17.552642\du}{-1.522124\du}}
\pgfusepath{stroke}
}
\pgfsetlinewidth{0.100000\du}
\pgfsetdash{}{0pt}
\pgfsetdash{}{0pt}
\pgfsetbuttcap
{
\definecolor{dialinecolor}{rgb}{0.000000, 0.000000, 0.000000}
\pgfsetfillcolor{dialinecolor}
% was here!!!
\pgfsetarrowsend{stealth}
\definecolor{dialinecolor}{rgb}{0.000000, 0.000000, 0.000000}
\pgfsetstrokecolor{dialinecolor}
\draw (7.518451\du,0.872365\du)--(7.642387\du,0.942767\du);
}
\pgfsetlinewidth{0.100000\du}
\pgfsetdash{}{0pt}
\pgfsetdash{}{0pt}
\pgfsetbuttcap
{
\definecolor{dialinecolor}{rgb}{0.000000, 0.000000, 0.000000}
\pgfsetfillcolor{dialinecolor}
% was here!!!
\pgfsetarrowsend{stealth}
\definecolor{dialinecolor}{rgb}{0.000000, 0.000000, 0.000000}
\pgfsetstrokecolor{dialinecolor}
\draw (12.349756\du,0.605352\du)--(12.478752\du,0.566730\du);
}
\pgfsetlinewidth{0.050000\du}
\pgfsetdash{}{0pt}
\pgfsetdash{}{0pt}
\pgfsetmiterjoin
\pgfsetbuttcap
{
\definecolor{dialinecolor}{rgb}{0.000000, 0.000000, 0.000000}
\pgfsetfillcolor{dialinecolor}
% was here!!!
\definecolor{dialinecolor}{rgb}{0.000000, 0.000000, 0.000000}
\pgfsetstrokecolor{dialinecolor}
\pgfpathmoveto{\pgfpoint{5.652241\du}{6.986376\du}}
\pgfpathcurveto{\pgfpoint{6.629568\du}{4.298725\du}}{\pgfpoint{9.719649\du}{1.740426\du}}{\pgfpoint{18.387167\du}{3.356812\du}}
\pgfusepath{stroke}
}
\pgfsetlinewidth{0.100000\du}
\pgfsetdash{}{0pt}
\pgfsetdash{}{0pt}
\pgfsetbuttcap
{
\definecolor{dialinecolor}{rgb}{0.000000, 0.000000, 0.000000}
\pgfsetfillcolor{dialinecolor}
% was here!!!
\pgfsetarrowsend{stealth}
\definecolor{dialinecolor}{rgb}{0.000000, 0.000000, 0.000000}
\pgfsetstrokecolor{dialinecolor}
\draw (14.844684\du,2.901757\du)--(14.971057\du,2.909975\du);
}
\pgfsetlinewidth{0.100000\du}
\pgfsetdash{}{0pt}
\pgfsetdash{}{0pt}
\pgfsetbuttcap
{
\definecolor{dialinecolor}{rgb}{0.000000, 0.000000, 0.000000}
\pgfsetfillcolor{dialinecolor}
% was here!!!
\pgfsetarrowsend{stealth}
\definecolor{dialinecolor}{rgb}{0.000000, 0.000000, 0.000000}
\pgfsetstrokecolor{dialinecolor}
\draw (7.918989\du,4.176805\du)--(8.021761\du,4.111354\du);
}
\end{tikzpicture}

\end{center}
\end{columns}
\end{frame}

\begin{frame}[fragile]
\frametitle{The structure of the phase space}
\begin{itemize}\itemsep1em
\item One important characteristic of the phase space is the set of points where the right-hand side of one of the equations in \eqref{eq:phsystem} vanishes.
\item These define (possibly complicated looking) curves that partition the phase space into regions via the equations $ F(x,\lambda)=0 $ and $ G(x,\lambda)=0 $. 
\item The curves correspond to positions where movement in one (or both) variable(s) stops.
\item Any point in the intersection of those sets, i.e. a solution $ (\bar{x},\bar{\lambda}) $ of the algebraic system \[ F(x,\lambda)=0,\qquad G(x,\lambda)=0, \] is called a \emph{stationary point} or an \emph{equilibrium point}.
\item An equilibrium point is also a (constant) solution of the system of differential equations \eqref{eq:phsystem}.
\end{itemize}
\end{frame}

\begin{frame}[fragile]
\frametitle{An example of a phase portrait (SS, p. 252)}
\begin{columns}
\column{.25\textwidth}
\textbf{Highlights:}
\begin{itemize}\itemsep1em
\item There are 2 curves ($ \dot{x}=0 $ and $ \dot{\lambda}=0 $) dividing the plane into 4 regions (I-IV)
\item Each region has a specific direction of the vector field
\end{itemize}
\column{.7\textwidth}
\begin{center}
% Graphic for TeX using PGF
% Title: S:\ANDREY\FEBA\Applied_Econometrics_Econ_Modelling_MSc\Optimization\SS3.dia
% Creator: Dia v0.97.2
% CreationDate: Sat Nov 19 11:54:56 2016
% For: User
% \usepackage{tikz}
% The following commands are not supported in PSTricks at present
% We define them conditionally, so when they are implemented,
% this pgf file will use them.
\ifx\du\undefined
  \newlength{\du}
\fi
\setlength{\du}{15\unitlength}
\begin{tikzpicture}
\pgftransformxscale{1.000000}
\pgftransformyscale{-1.000000}
\definecolor{dialinecolor}{rgb}{0.000000, 0.000000, 0.000000}
\pgfsetstrokecolor{dialinecolor}
\definecolor{dialinecolor}{rgb}{1.000000, 1.000000, 1.000000}
\pgfsetfillcolor{dialinecolor}
\pgfsetlinewidth{0.030000\du}
\pgfsetdash{}{0pt}
\pgfsetdash{}{0pt}
\pgfsetbuttcap
{
\definecolor{dialinecolor}{rgb}{0.000000, 0.000000, 0.000000}
\pgfsetfillcolor{dialinecolor}
% was here!!!
\pgfsetarrowsend{stealth}
\definecolor{dialinecolor}{rgb}{0.000000, 0.000000, 0.000000}
\pgfsetstrokecolor{dialinecolor}
\draw (5.000000\du,5.000000\du)--(20.027400\du,4.978920\du);
}
\pgfsetlinewidth{0.030000\du}
\pgfsetdash{}{0pt}
\pgfsetdash{}{0pt}
\pgfsetbuttcap
{
\definecolor{dialinecolor}{rgb}{0.000000, 0.000000, 0.000000}
\pgfsetfillcolor{dialinecolor}
% was here!!!
\pgfsetarrowsend{stealth}
\definecolor{dialinecolor}{rgb}{0.000000, 0.000000, 0.000000}
\pgfsetstrokecolor{dialinecolor}
\draw (9.000000\du,7.600000\du)--(9.000000\du,-4.000000\du);
}
% setfont left to latex
\definecolor{dialinecolor}{rgb}{0.000000, 0.000000, 0.000000}
\pgfsetstrokecolor{dialinecolor}
\node[anchor=west] at (7.681980\du,-3.8\du){$ \lambda $};
% setfont left to latex
\definecolor{dialinecolor}{rgb}{0.000000, 0.000000, 0.000000}
\pgfsetstrokecolor{dialinecolor}
\node[anchor=west] at (19.918800\du,5.645820\du){$ x $};
\pgfsetlinewidth{0.060000\du}
\pgfsetdash{}{0pt}
\pgfsetdash{}{0pt}
\pgfsetmiterjoin
\pgfsetbuttcap
{
\definecolor{dialinecolor}{rgb}{0.000000, 0.000000, 0.000000}
\pgfsetfillcolor{dialinecolor}
% was here!!!
\definecolor{dialinecolor}{rgb}{0.000000, 0.000000, 0.000000}
\pgfsetstrokecolor{dialinecolor}
\pgfpathmoveto{\pgfpoint{5.278560\du}{-2.212000\du}}
\pgfpathcurveto{\pgfpoint{6.845160\du}{0.619374\du}}{\pgfpoint{10.136450\du}{3.939413\du}}{\pgfpoint{18.343128\du}{1.467349\du}}
\pgfusepath{stroke}
}
\pgfsetlinewidth{0.060000\du}
\pgfsetdash{}{0pt}
\pgfsetdash{}{0pt}
\pgfsetmiterjoin
\pgfsetbuttcap
{
\definecolor{dialinecolor}{rgb}{0.000000, 0.000000, 0.000000}
\pgfsetfillcolor{dialinecolor}
% was here!!!
\definecolor{dialinecolor}{rgb}{0.000000, 0.000000, 0.000000}
\pgfsetstrokecolor{dialinecolor}
\pgfpathmoveto{\pgfpoint{10.783211\du}{-2.944998\du}}
\pgfpathcurveto{\pgfpoint{10.711349\du}{-0.386699\du}}{\pgfpoint{11.933008\du}{5.304797\du}}{\pgfpoint{18.285638\du}{6.713299\du}}
\pgfusepath{stroke}
}
% setfont left to latex
\definecolor{dialinecolor}{rgb}{0.000000, 0.000000, 0.000000}
\pgfsetstrokecolor{dialinecolor}
\node[anchor=west] at (17.480780\du,0.903230\du){$ \dot{x} = 0 $};
% setfont left to latex
\definecolor{dialinecolor}{rgb}{0.000000, 0.000000, 0.000000}
\pgfsetstrokecolor{dialinecolor}
\node[anchor=west] at (17\du,7.428329\du){$ \dot{\lambda} = 0 $};
\pgfsetlinewidth{0.030000\du}
\pgfsetdash{}{0pt}
\pgfsetdash{}{0pt}
\pgfsetbuttcap
{
\definecolor{dialinecolor}{rgb}{0.000000, 0.000000, 0.000000}
\pgfsetfillcolor{dialinecolor}
% was here!!!
\pgfsetarrowsend{stealth}
\definecolor{dialinecolor}{rgb}{0.000000, 0.000000, 0.000000}
\pgfsetstrokecolor{dialinecolor}
\draw (9.654546\du,-0.444028\du)--(9.654187\du,0.554634\du);
}
\pgfsetlinewidth{0.030000\du}
\pgfsetdash{}{0pt}
\pgfsetdash{}{0pt}
\pgfsetbuttcap
{
\definecolor{dialinecolor}{rgb}{0.000000, 0.000000, 0.000000}
\pgfsetfillcolor{dialinecolor}
% was here!!!
\pgfsetarrowsend{stealth}
\definecolor{dialinecolor}{rgb}{0.000000, 0.000000, 0.000000}
\pgfsetstrokecolor{dialinecolor}
\draw (9.640146\du,-0.444028\du)--(10.640146\du,-0.444028\du);
}
\pgfsetlinewidth{0.030000\du}
\pgfsetdash{}{0pt}
\pgfsetdash{}{0pt}
\pgfsetbuttcap
{
\definecolor{dialinecolor}{rgb}{0.000000, 0.000000, 0.000000}
\pgfsetfillcolor{dialinecolor}
% was here!!!
\pgfsetarrowsend{stealth}
\definecolor{dialinecolor}{rgb}{0.000000, 0.000000, 0.000000}
\pgfsetstrokecolor{dialinecolor}
\draw (13.588290\du,0.074818\du)--(13.588290\du,-0.925182\du);
}
\pgfsetlinewidth{0.030000\du}
\pgfsetdash{}{0pt}
\pgfsetdash{}{0pt}
\pgfsetbuttcap
{
\definecolor{dialinecolor}{rgb}{0.000000, 0.000000, 0.000000}
\pgfsetfillcolor{dialinecolor}
% was here!!!
\pgfsetarrowsend{stealth}
\definecolor{dialinecolor}{rgb}{0.000000, 0.000000, 0.000000}
\pgfsetstrokecolor{dialinecolor}
\draw (13.573890\du,0.074818\du)--(14.573890\du,0.074818\du);
}
\pgfsetlinewidth{0.030000\du}
\pgfsetdash{}{0pt}
\pgfsetdash{}{0pt}
\pgfsetbuttcap
{
\definecolor{dialinecolor}{rgb}{0.000000, 0.000000, 0.000000}
\pgfsetfillcolor{dialinecolor}
% was here!!!
\pgfsetarrowsend{stealth}
\definecolor{dialinecolor}{rgb}{0.000000, 0.000000, 0.000000}
\pgfsetstrokecolor{dialinecolor}
\draw (17.278850\du,5.250747\du)--(17.278850\du,4.250747\du);
}
\pgfsetlinewidth{0.030000\du}
\pgfsetdash{}{0pt}
\pgfsetdash{}{0pt}
\pgfsetbuttcap
{
\definecolor{dialinecolor}{rgb}{0.000000, 0.000000, 0.000000}
\pgfsetfillcolor{dialinecolor}
% was here!!!
\pgfsetarrowsend{stealth}
\definecolor{dialinecolor}{rgb}{0.000000, 0.000000, 0.000000}
\pgfsetstrokecolor{dialinecolor}
\draw (17.292046\du,5.235970\du)--(16.149986\du,5.232935\du);
}
\pgfsetlinewidth{0.030000\du}
\pgfsetdash{}{0pt}
\pgfsetdash{}{0pt}
\pgfsetbuttcap
{
\definecolor{dialinecolor}{rgb}{0.000000, 0.000000, 0.000000}
\pgfsetfillcolor{dialinecolor}
% was here!!!
\pgfsetarrowsend{stealth}
\definecolor{dialinecolor}{rgb}{0.000000, 0.000000, 0.000000}
\pgfsetstrokecolor{dialinecolor}
\draw (10.807550\du,4.391780\du)--(9.725645\du,4.392449\du);
}
\pgfsetlinewidth{0.030000\du}
\pgfsetdash{}{0pt}
\pgfsetdash{}{0pt}
\pgfsetbuttcap
{
\definecolor{dialinecolor}{rgb}{0.000000, 0.000000, 0.000000}
\pgfsetfillcolor{dialinecolor}
% was here!!!
\pgfsetarrowsend{stealth}
\definecolor{dialinecolor}{rgb}{0.000000, 0.000000, 0.000000}
\pgfsetstrokecolor{dialinecolor}
\draw (10.792640\du,4.391280\du)--(10.793206\du,5.693425\du);
}
% setfont left to latex
\definecolor{dialinecolor}{rgb}{0.000000, 0.000000, 0.000000}
\pgfsetstrokecolor{dialinecolor}
\node[anchor=west] at (15.704440\du,-1.723338\du){I};
% setfont left to latex
\definecolor{dialinecolor}{rgb}{0.000000, 0.000000, 0.000000}
\pgfsetstrokecolor{dialinecolor}
\node[anchor=west] at (7.339665\du,-1.608359\du){II};
% setfont left to latex
\definecolor{dialinecolor}{rgb}{0.000000, 0.000000, 0.000000}
\pgfsetstrokecolor{dialinecolor}
\node[anchor=west] at (7.440272\du,4.226862\du){III};
% setfont left to latex
\definecolor{dialinecolor}{rgb}{0.000000, 0.000000, 0.000000}
\pgfsetstrokecolor{dialinecolor}
\node[anchor=west] at (15.072051\du,3.651964\du){IV};
% setfont left to latex
\definecolor{dialinecolor}{rgb}{0.000000, 0.000000, 0.000000}
\pgfsetstrokecolor{dialinecolor}
\node[anchor=west] at (12.0\du,1.6\du){$ E = (\bar{x},\bar{\lambda}) $};
\end{tikzpicture}

\end{center}
\end{columns}
\end{frame}

\begin{frame}[fragile]
\frametitle{The same phase portrait with the corresponding solution trajectories (SS, p. 252)}
\begin{center}
\ifx\du\undefined
  \newlength{\du}
\fi
\setlength{\du}{15\unitlength}
\begin{tikzpicture}
\pgftransformxscale{1.000000}
\pgftransformyscale{-1.000000}
\definecolor{dialinecolor}{rgb}{0.000000, 0.000000, 0.000000}
\pgfsetstrokecolor{dialinecolor}
\definecolor{dialinecolor}{rgb}{1.000000, 1.000000, 1.000000}
\pgfsetfillcolor{dialinecolor}
\pgfsetlinewidth{0.030000\du}
\pgfsetdash{}{0pt}
\pgfsetdash{}{0pt}
\pgfsetbuttcap
{
\pgfsetarrowsend{stealth}
\definecolor{dialinecolor}{rgb}{0.000000, 0.000000, 0.000000}
\pgfsetstrokecolor{dialinecolor}
\draw (5.000000\du,5.000000\du)--(20.000000\du,5.000000\du);
}
\pgfsetlinewidth{0.030000\du}
\pgfsetdash{}{0pt}
\pgfsetdash{}{0pt}
\pgfsetbuttcap
{
\pgfsetarrowsend{stealth}
\definecolor{dialinecolor}{rgb}{0.000000, 0.000000, 0.000000}
\pgfsetstrokecolor{dialinecolor}
\draw (9.000000\du,7.600000\du)--(9.000000\du,-4.000000\du);
}
% setfont left to latex
\definecolor{dialinecolor}{rgb}{0.000000, 0.000000, 0.000000}
\pgfsetstrokecolor{dialinecolor}
\node[anchor=west] at (7.681980\du,-3.8\du){$ \lambda $};
% setfont left to latex
\definecolor{dialinecolor}{rgb}{0.000000, 0.000000, 0.000000}
\pgfsetstrokecolor{dialinecolor}
\node[anchor=west] at (19.918800\du,5.645820\du){$ x $};
\pgfsetlinewidth{0.060000\du}
\pgfsetdash{}{0pt}
\pgfsetdash{}{0pt}
\pgfsetmiterjoin
\pgfsetbuttcap
{
\definecolor{dialinecolor}{rgb}{0.000000, 0.000000, 0.000000}
\pgfsetstrokecolor{dialinecolor}
\pgfpathmoveto{\pgfpoint{5.278560\du}{-2.212000\du}}
\pgfpathcurveto{\pgfpoint{6.845160\du}{0.619374\du}}{\pgfpoint{10.136500\du}{3.939410\du}}{\pgfpoint{18.343100\du}{1.467350\du}}
\pgfusepath{stroke}
}
\pgfsetlinewidth{0.060000\du}
\pgfsetdash{}{0pt}
\pgfsetdash{}{0pt}
\pgfsetmiterjoin
\pgfsetbuttcap
{
\definecolor{dialinecolor}{rgb}{0.000000, 0.000000, 0.000000}
\pgfsetstrokecolor{dialinecolor}
\pgfpathmoveto{\pgfpoint{10.783200\du}{-2.945000\du}}
\pgfpathcurveto{\pgfpoint{10.711300\du}{-0.386699\du}}{\pgfpoint{11.933000\du}{5.304800\du}}{\pgfpoint{18.285600\du}{6.713300\du}}
\pgfusepath{stroke}
}
% setfont left to latex
\definecolor{dialinecolor}{rgb}{0.000000, 0.000000, 0.000000}
\pgfsetstrokecolor{dialinecolor}
\node[anchor=west] at (17.480800\du,0.903230\du){$ \dot{x} = 0 $};
% setfont left to latex
\definecolor{dialinecolor}{rgb}{0.000000, 0.000000, 0.000000}
\pgfsetstrokecolor{dialinecolor}
\node[anchor=west] at (17.207702\du,7.428330\du){$ \dot{\lambda} = 0 $};
% setfont left to latex
\definecolor{dialinecolor}{rgb}{0.000000, 0.000000, 0.000000}
\pgfsetstrokecolor{dialinecolor}
\node[anchor=west] at (12.5\du,2.0\du){$ E $};
\pgfsetlinewidth{0.040000\du}
\pgfsetdash{}{0pt}
\pgfsetdash{}{0pt}
\pgfsetmiterjoin
\pgfsetbuttcap
{
\definecolor{dialinecolor}{rgb}{0.000000, 0.000000, 0.000000}
\pgfsetstrokecolor{dialinecolor}
\pgfpathmoveto{\pgfpoint{8.777222\du}{-3.112355\du}}
\pgfpathcurveto{\pgfpoint{10.293429\du}{-0.683480\du}}{\pgfpoint{11.956840\du}{-1.095653\du}}{\pgfpoint{13.355283\du}{-2.611859\du}}
\pgfusepath{stroke}
}
\pgfsetlinewidth{0.100000\du}
\pgfsetdash{}{0pt}
\pgfsetdash{}{0pt}
\pgfsetbuttcap
{
\pgfsetarrowsend{stealth}
\definecolor{dialinecolor}{rgb}{0.000000, 0.000000, 0.000000}
\pgfsetstrokecolor{dialinecolor}
\draw (7.155876\du,-0.830019\du)--(7.183277\du,-0.682354\du);
}
\pgfsetlinewidth{0.100000\du}
\pgfsetdash{}{0pt}
\pgfsetdash{}{0pt}
\pgfsetbuttcap
{
\pgfsetarrowsend{stealth}
\definecolor{dialinecolor}{rgb}{0.000000, 0.000000, 0.000000}
\pgfsetstrokecolor{dialinecolor}
\draw (12.231914\du,-1.701150\du)--(12.366125\du,-1.766645\du);
}
\pgfsetlinewidth{0.040000\du}
\pgfsetdash{}{0pt}
\pgfsetdash{}{0pt}
\pgfsetmiterjoin
\pgfsetbuttcap
{
\definecolor{dialinecolor}{rgb}{0.000000, 0.000000, 0.000000}
\pgfsetstrokecolor{dialinecolor}
\pgfpathmoveto{\pgfpoint{11.857346\du}{6.928886\du}}
\pgfpathcurveto{\pgfpoint{12.855573\du}{4.902368\du}}{\pgfpoint{15.121737\du}{4.821969\du}}{\pgfpoint{18.124710\du}{5.955444\du}}
\pgfusepath{stroke}
}
\pgfsetlinewidth{0.040000\du}
\pgfsetdash{}{0pt}
\pgfsetdash{}{0pt}
\pgfsetmiterjoin
\pgfsetbuttcap
{
\definecolor{dialinecolor}{rgb}{0.000000, 0.000000, 0.000000}
\pgfsetstrokecolor{dialinecolor}
\pgfpathmoveto{\pgfpoint{6.467671\du}{-2.513824\du}}
\pgfpathcurveto{\pgfpoint{7.344391\du}{-1.493379\du}}{\pgfpoint{7.962407\du}{1.783543\du}}{\pgfpoint{5.504716\du}{3.551357\du}}
\pgfusepath{stroke}
}
\pgfsetlinewidth{0.040000\du}
\pgfsetdash{}{0pt}
\pgfsetdash{}{0pt}
\pgfsetmiterjoin
\pgfsetbuttcap
{
\definecolor{dialinecolor}{rgb}{255.000000, 0.000000, 0.000000}
\pgfsetstrokecolor{dialinecolor}
\pgfpathmoveto{\pgfpoint{9.514634\du}{-0.487306\du}}
\pgfpathcurveto{\pgfpoint{11.167467\du}{1.970385\du}}{\pgfpoint{12.466815\du}{3.175733\du}}{\pgfpoint{18.023133\du}{4.384959\du}}
\pgfusepath{stroke}
}
\pgfsetlinewidth{0.040000\du}
\pgfsetdash{}{0pt}
\pgfsetdash{}{0pt}
\pgfsetmiterjoin
\pgfsetbuttcap
{
\definecolor{dialinecolor}{rgb}{0.000000, 0.000000, 0.000000}
\pgfsetstrokecolor{dialinecolor}
\pgfpathmoveto{\pgfpoint{7.442980\du}{5.920870\du}}
\pgfpathcurveto{\pgfpoint{9.814437\du}{5.374716\du}}{\pgfpoint{13.508161\du}{1.134839\du}}{\pgfpoint{14.183667\du}{-1.164756\du}}
\pgfusepath{stroke}
}
\pgfsetlinewidth{0.100000\du}
\pgfsetdash{}{0pt}
\pgfsetdash{}{0pt}
\pgfsetbuttcap
{
\pgfsetarrowsend{stealth}
\definecolor{dialinecolor}{rgb}{0.000000, 0.000000, 0.000000}
\pgfsetstrokecolor{dialinecolor}
\draw (6.961834\du,1.634318\du)--(6.913591\du,1.778691\du);
}
\pgfsetlinewidth{0.100000\du}
\pgfsetdash{}{0pt}
\pgfsetdash{}{0pt}
\pgfsetbuttcap
{
\pgfsetarrowsend{stealth}
\definecolor{dialinecolor}{rgb}{0.000000, 0.000000, 0.000000}
\pgfsetstrokecolor{dialinecolor}
\draw[red] (10.268347\du,0.555975\du)--(10.351024\du,0.661307\du);
}
\pgfsetlinewidth{0.100000\du}
\pgfsetdash{}{0pt}
\pgfsetdash{}{0pt}
\pgfsetbuttcap
{
\pgfsetarrowsend{stealth}
\definecolor{dialinecolor}{rgb}{0.000000, 0.000000, 0.000000}
\pgfsetstrokecolor{dialinecolor}
\draw[red] (14.628824\du,3.470003\du)--(14.518858\du,3.433733\du);
}
\pgfsetlinewidth{0.100000\du}
\pgfsetdash{}{0pt}
\pgfsetdash{}{0pt}
\pgfsetbuttcap
{
\pgfsetarrowsend{stealth}
\definecolor{dialinecolor}{rgb}{0.000000, 0.000000, 0.000000}
\pgfsetstrokecolor{dialinecolor}
\draw (13.427773\du,0.440696\du)--(13.508860\du,0.310567\du);
}
\pgfsetlinewidth{0.100000\du}
\pgfsetdash{}{0pt}
\pgfsetdash{}{0pt}
\pgfsetbuttcap
{
\pgfsetarrowsend{stealth}
\definecolor{dialinecolor}{rgb}{0.000000, 0.000000, 0.000000}
\pgfsetstrokecolor{dialinecolor}
\draw (10.392643\du,4.086752\du)--(10.282568\du,4.190849\du);
}
\pgfsetlinewidth{0.100000\du}
\pgfsetdash{}{0pt}
\pgfsetdash{}{0pt}
\pgfsetbuttcap
{
\pgfsetarrowsend{stealth}
\definecolor{dialinecolor}{rgb}{0.000000, 0.000000, 0.000000}
\pgfsetstrokecolor{dialinecolor}
\draw (16.326914\du,5.408017\du)--(16.168920\du,5.372544\du);
}
\pgfsetlinewidth{0.100000\du}
\pgfsetdash{}{0pt}
\pgfsetdash{}{0pt}
\pgfsetbuttcap
{
\pgfsetarrowsend{stealth}
\definecolor{dialinecolor}{rgb}{0.000000, 0.000000, 0.000000}
\pgfsetstrokecolor{dialinecolor}
\draw (12.590445\du,5.965373\du)--(12.492593\du,6.050547\du);
}
\end{tikzpicture}

\end{center}
\end{frame}

\begin{frame}[fragile]
\frametitle{Characteristics of the phase portrait}
\begin{itemize}\itemsep1em
\item The last phase portrait illustrates a situation typical of many economic models.
\item The equilibrium point $ E $ is of a special kind called a \emph{saddle point}.
\item There are only two trajectories (the red lines) that converge to the equilibrium.
\item Notice that in an optimal control context these curves will be natural candidates to check in order to verify that the transversality condition holds.
\item Unfortunately the phase portrait is only suggestive and the nature of the equilibrium point has to be established through formal means.
\item The next result provides a more precise characterization.
\end{itemize}
\end{frame}

\begin{frame}[fragile]
\frametitle{A local saddle point result}
\begin{Fact}
Consider the system \eqref{eq:phsystem} and take an equilibrium point of this system $ (\bar{x},\bar{\lambda}) $. Compute the Jacobian of the system and evaluate it at the equilibrium point:
\[ \begin{pmatrix}
F'_x(\bar{x},\bar{\lambda})& F'_{\lambda}(\bar{x},\bar{\lambda})\\
G'_x(\bar{x},\bar{\lambda})& G'_{\lambda}(\bar{x},\bar{\lambda})
\end{pmatrix} .\]

Let the eigenvalues of the above matrix be real and of opposite signs, i.e. \[ (F'_x + G'_\lambda)^2-4(F'_x G'_\lambda - F'_{\lambda} G'_x) >0,\qquad F'_x G'_\lambda - F'_{\lambda} G'_x < 0. \]

Then $ (\bar{x},\bar{\lambda}) $ is a \emph{saddle point}, which means that there exist exactly two solutions $ (x_1(t),\lambda_1(t)) $ and $ (x_2(t),\lambda_2(t)) $ defined on an interval $ [t_0,\infty) $ and converging to $ (\bar{x},\bar{\lambda}) $ from opposite directions. These solutions are tangent to a line through $ (\bar{x},\bar{\lambda}) $ that has the same direction as the eigenvector corresponding to the negative eigenvalue.
\label{fc:locsaddlept}
\end{Fact}
\end{frame}

\begin{frame}[fragile]
\frametitle{Further comments}
\begin{itemize}\itemsep1em
\item Fact \ref{fc:locsaddlept} is useful as a clean characterization of a saddle point but it might prove too restrictive for optimal control applications because of its local nature.
\item The risk is that the two solutions might be defined only in a small neighbourhood of the saddle point. However, we only have one degree of freedom (for the initial $ \lambda $) while the initial value of $ x $ is fixed at $ x_0 $.
\item Fortunately there exist global saddle point results which overcome this deficiency. See SS, p. 256.
\item While we framed the results in $ x\lambda $ space, sometimes it proves useful to replace the adjoint equation with a suitable differential equation for the control $ u $ derived from the maximum condition and study the resulting system in $ xu $ space.
\end{itemize}
\end{frame}

\end{section}

\begin{frame}[fragile]
\frametitle{Readings}
Main references:\bigskip

Seierstad and Syds\ae{}ter [SS]. \emph{Optimal control theory with economic applications}. Chapter 3.

Syds\ae{}ter et al. [SHSS] \emph{Further mathematics for economic analysis}. Chapter 9.\bigskip

Additional readings:

Sethi and Thompson [ST]. \emph{Optimal control theory: applications to management science and economics}. Chapter 3.
\end{frame}

\end{document}

