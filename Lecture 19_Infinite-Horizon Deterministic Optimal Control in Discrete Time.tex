% !TeX spellcheck = en_GB
\documentclass[10pt]{beamer}
\usetheme{CambridgeUS}
%\usetheme{Boadilla}
\definecolor{myred}{RGB}{163,0,0}
%\usecolortheme[named=blue]{structure}
\usecolortheme{dove}
\usefonttheme[]{professionalfonts}
\usepackage[english]{babel}
\usepackage{amsmath,amsfonts,amssymb}
\usepackage{xcolor}
\usepackage{bm}
\usepackage{gensymb}
\usepackage{verbatim} 
\usepackage{paratype}
\usepackage{mathpazo}
\usepackage{listings}
\lstset{language=Python}

\usepackage{tikz}
\usetikzlibrary{matrix}

\DeclareMathOperator*{\interior}{int}

% Number theorem environments
\setbeamertemplate{theorem}[ams style]
\setbeamertemplate{theorems}[numbered]

% Reset theorem-like environments so that each is numbered separately
\usepackage{etoolbox}
\undef{\definition}
\theoremstyle{definition}
\newtheorem{definition}{\translate{Definition}}
\newtheorem{Fact}{\translate{Fact}}

% Change colours for theorem-like environments
\definecolor{mygreen1}{RGB}{0,96,0}
\definecolor{mygreen2}{RGB}{229,239,229}
\setbeamercolor{block title}{fg=white,bg=mygreen1}
\setbeamercolor{block body}{fg=black,bg=mygreen2}



\alt<presentation>
{\lstset{%
  basicstyle=\footnotesize\ttfamily,
  commentstyle=\slshape\color{green!50!black},
  frame = single,  
  keywordstyle=\bfseries\color{blue!50!black},
  identifierstyle=\color{blue},
  stringstyle=\color{orange},
  %escapechar=\#,
  showstringspaces = false,
  showtabs = false,
  tabsize = 2,
  emphstyle=\color{red}}
}
{
  \lstset{%
    basicstyle=\ttfamily,
    keywordstyle=\bfseries,
    commentstyle=\itshape,
    escapechar=\#,
    showtabs = false,
	tabsize = 2,
    emphstyle=\bfseries\color{red}
  }
} 

\title{R401: Statistical and Mathematical Foundations \bigskip}
\subtitle{\textcolor{myred}{Infinite-Horizon Deterministic Optimal Control in Discrete Time}}
\author{Andrey Vassilev}

\date{} 
    
\AtBeginSection{\frame{\usebeamerfont{section title}\centering\insertsection}}

\begin{document}
\maketitle



\begin{frame}[fragile]
\frametitle{Lecture Contents}
\tableofcontents
\end{frame}

\begin{section}{Introduction}\label{sec:intro}

\begin{frame}[fragile]
\frametitle{Switching to discrete time}
\begin{itemize}\itemsep1em
\item You are already familiar with a number of dynamic optimization problems in which time is continuous.
\item In many applications, however, it is natural to work in discrete time.
\item This provides a bridge to validating/calibrating models with data or estimating them, among others.
\item We therefore need to develop the counterpart of the continuous-time optimal control framework for the case of discrete time.
\end{itemize}
\end{frame}

\begin{frame}[fragile]
\frametitle{Switching to discrete time}
\begin{itemize}\itemsep1em
\item Some of the details of such a transition are easily predictable:
	\begin{itemize}
	\item Differential equations will be replaced by difference equations.
	\item The objective functional will involve a series instead of an integral.
	\end{itemize}
\item Other details need to be specified further. In particular, there exist two broad classes of dynamic optimization problems in discrete time:
	\begin{itemize}
	\item Problems in variational form
	\item Problems with explicit controls
	\end{itemize}
\item Both classes can be used to address a wide variety of problems. 
\item However, problems with explicit controls are a bit more transparent in terms of their structure.
\end{itemize}
\end{frame}

\begin{frame}[fragile]
\frametitle{Specific assumptions}
\begin{itemize}\itemsep1em
\item We will work in an infinite-horizon setup. Finite-horizon formulations for discrete-time problems exist but are less common in economic applications.
\item We sacrifice some generality from the outset by assuming a specific structure of the problems:
	\begin{itemize}
	\item Special (time) discounting in the objective functionals.
	\item Autonomous difference equations describing the evolution of the system that is being modelled.
	\end{itemize}
\end{itemize}
\end{frame}
\end{section}


\begin{section}{Problems in variational form}\label{sec:varprob}

\begin{frame}[fragile]
\frametitle{Formulation}
A dynamic optimization problem in variational form is defined as follows:
\begin{equation}\label{eq:OCvarform} 
\begin{split}
 \max&_{\{x_{t+1}\}_{t=0}^\infty} \sum_{t=0}^{\infty}\beta^t F(x_t,x_{t+1})\\
\text{s.t. } & x_{t+1}\in \Gamma(x_t),\quad t=0,1,2\ldots, \\
& x_0\in X \text{ -- given}
\end{split}
\end{equation}

The problem is characterized by the following:
\begin{itemize}
\item We choose directly the sequence $ \{x_t\}_{t=1}^\infty $. For any element $ x_t $ we have $ x_t\in X $, where $ X $ is the set of states.
\item At any point in time $ x_t $ defines a set $ \Gamma(x_t) $ of admissible values for $ x $ in the following period. 
\item The number $ \beta $ is called  the \emph{discount factor} and $ \beta \in (0,1) $.
\item We shall assume differentiability of the function $ F $ as needed.
\item We write $ \max $ everywhere with some sacrifice of mathematical precision.
\end{itemize}
\end{frame}

\begin{frame}[fragile]
\frametitle{Additional information on problems in variational form}
\framesubtitle{The Bellman equation}
\begin{itemize}\itemsep1em
\item A problem of the form given in \eqref{eq:OCvarform} has an associated equation of the form \begin{equation}\label{eq:BellmanVarOC}
v(x) = \max_{y\in \Gamma(x)}\left \{ F(x,y)+\beta v(y)  \right \},\quad \forall x \in X.
\end{equation}
\item  The equation is called the \emph{Bellman equation}.
\item  The Bellman equation is a \emph{functional} equation: it involves finding an unknown function $ v $.
\item The Bellman equation may not have a solution or it may have multiple solutions.
\item Notice that solving the Bellman equation (however it may be done) involves finding the maximizing value $ y^*\in \Gamma(x) $.
\end{itemize}
\end{frame}

\begin{frame}[fragile]
\frametitle{Additional information on problems in variational form}
\framesubtitle{The value function}
\begin{itemize}\itemsep1em
\item Denote the set of all feasible sequences $  \{x_t\}_{t=0}^{\infty}$ starting from $ x_0 $ by $ \Pi(x_0) $.
\item Define the function $ v^*(x_0) $ as \[ v^*(x_0):= \max_{\{x_t\}\in \Pi(x_0)}\sum_{t=0}^{\infty}\beta^t F(x_t,x_{t+1}). \]
\item This function is known as the \emph{value function}.
\item It can be shown that the value function is (one) solution to the Bellman equation.
\item Moreover, $ v^* $ is the only solution to the Bellman equation that satisfies the boundedness condition \[ \lim\limits_{t\rightarrow \infty} \beta^t v^*(x_t) =0 \text{ for all }(x_0,x_1,\ldots)\in \Pi(x_0)\text{ and all }x_0\in X. \]
\end{itemize}
\end{frame}

\begin{frame}[fragile]
\frametitle{Additional information on problems in variational form}
\begin{itemize}\itemsep1em
\item It can also be shown that an optimal sequence $ \{x_t\} $ for problem \eqref{eq:OCvarform} satisfies the relations \[ v^*(x_t^*) = F(x_t^*,x_{t+1}^*)+\beta v^*(x_{t+1}^*),\quad t=0,1,2,\ldots \] \pause
\item \alert{The above statements are only indicative. The precise formulations require specific assumptions on the mathematical structure of the problem.} \pause
\item The Bellman equation approach is quite general. However, it often leads to situations which are analytically intractable and computationally demanding.
\item For this reason it is typical to resort to more restrictive but tractable approaches.
\end{itemize}
\end{frame}


\begin{frame}[fragile]
\frametitle{The Euler equations}
\framesubtitle{}
\begin{itemize}\itemsep1em
\item One approach that is essentially the discrete-time counterpart of the Euler equation from the calculus of variations can be used to provide necessary conditions for optimality.
\item Because of this similarity, the resulting necessary conditions are also called \emph{Euler equations}.
\item They take the form
\begin{equation}\label{eq:Euler}
\nabla_y F(x_t,x_{t+1})+\beta \nabla_x F(x_{t+1},x_{t+2})=0.
\end{equation}
\item Notice that they lead to a second-order difference equation (or, more precisely, a system of second-order difference equations), just like the Euler equation for the continuous-time variational problem produced a second-order ODE.
\end{itemize}
\end{frame}

\begin{frame}[fragile]
\frametitle{Sufficiency}
\framesubtitle{}
\begin{itemize}\itemsep1em
\item The Euler equations can be supplemented with appropriate transversality conditions to obtain sufficient conditions for optimality.
\item The precise formulation requires technical concepts and assumptions that are beyond this course (see Stokey and Lucas, Ch. 4).
\item The main assumptions are that $ X $ is a convex subset of $ \mathbb{R}^n_{+} $, $ \Gamma (x) $ is nonempty and compact, $ F $ is bounded, concave, differentiable and strictly increasing in $ x_t $. There are additional assumptions and qualifications.
\item The essence of the sufficiency result is that, under the required assumptions, a feasible sequence $ \{x^*_t\}_{t=0}^{\infty} $ satisfying the Euler equations \eqref{eq:Euler} and the transversality condition \[ \lim\limits_{t\rightarrow \infty}\beta^t \nabla_x F(x_t^*,x_{t+1}^*) x^*_t = 0 \] is optimal for problem \eqref{eq:OCvarform}.
\end{itemize}


\end{frame}



\begin{frame}[fragile]
\frametitle{Example: NCs for a problem in variational form}
Compute the NCs for the problem:
\[ \max_{\{c_t\}} \sum_{t=0}^{\infty}\beta^t \ln c_t  \]
\[ k_{t+1} = (1-\delta)k_t + y_t - c_t,\quad k_0>0 \text{ -- given} \] \[ y_t=A k_t^\alpha , \quad 0<\underline{\epsilon} \leq c_t \leq \overline{\epsilon}, \quad \alpha,\beta \in (0,1), \quad A>0\]\bigskip \pause

To write the NCs we transform the problem in variational form: 
\[ \sum_{t=0}^{\infty}\beta^t \underbrace{\ln ((1-\delta)k_t + A k_t^\alpha - k_{t+1})}_{= F(x_t,x_{t+1})}. \]
\end{frame}



\begin{frame}[fragile]
\frametitle{Example: NCs for a problem in variational form}
Applying the Euler equation, we get:
\[ \frac{-1}{(1-\delta)k_t +A k_t^\alpha -k_{t+1}} + \beta \frac{(1-\delta) + \alpha A k_{t+1}^{\alpha -1}}{(1-\delta)k_{t+1} +A k_{t+1}^\alpha -k_{t+2}} = 0. \]

The last result has the form 
\[ G(x_{t+2},x_{t+1},x_t)=0, \]
i.e. a nonlinear second-order difference equation.

\pause \bigskip
\textbf{Question:} Can you compute the solution of the equation for $ \beta = 0.95,~A=1,~ \delta = 0.05,~ \alpha = 0.5 $ and $ k_0 = 2 $? \pause If not, what else do you need?
\end{frame}

\end{section}

\begin{section}{Problems with explicit controls}\label{sec:explcontr}

\begin{frame}[fragile]
\frametitle{Formulation of the problem with explicit controls}
It is also possible to introduce controls explicitly, as a direct counterpart of the continuous-time formulation. This is done as follows.\bigskip

Let $X \subset \mathbb{R}^n$ be the \emph{state space} for a model, where the state variables are $x=(x^1,\ldots,x^n)$. \bigskip

We assume that $\forall x \in X,~ \exists \Omega(x) \subset \mathbb{R}^m, \Omega(x)\neq \emptyset$. The elements of $u=(u^1,\ldots,u^m)$ are our \emph{controls}. \bigskip

The (instantaneous) objective function is $F(x,u)$ за $x \in X$, $u \in \Omega(x)$ \bigskip

The state equations are
\begin{equation}
x_{t+1}=f(x_t,u_t),\quad x_0 \text{ -- given} \label{eq:state}
\end{equation}
where $f(x,u)$ is a vector function taking values in $X$, for $x \in X$, $u \in \Omega(x)$.\bigskip
\end{frame}



\begin{frame}[fragile]
\frametitle{Formulation of the problem with explicit controls}

We need to find a sequence of admissible controls $\mathbf{u} = \{u_t\}$, $t=0,1,2,\ldots,$
which determine a sequence of state variables $\{ x_{t+1} \}$, $t=0,1,2,\ldots$ via \eqref{eq:state} for which \begin{equation}
J(x_0,\mathbf{u})=\sum_{t=0}^\infty
\beta^t F(x_t,u_t)\label{eq:obj}
\end{equation} attains a maximum \begin{equation}
v(x_0)=\max_{\mathbf{u}} J(x_0,\mathbf{u}).  
\label{eq:objMax}
\end{equation} 
\end{frame}



\begin{frame}[fragile]
\frametitle{Formulation of the problem with explicit controls}
The number $\beta \in (0,1)$ is the \emph{discount factor} in the model. \bigskip

Denote by $\textrm{FC}(x_0)$ the set of all feasible control sequences $\{ u_t\}_{t=0}^\infty$ for initial $x_0 \in X$, i.e. $ x_{t+1} $
satisfies \eqref{eq:state} for $ u_t \in \Omega(x_t)$, $t=0,1,2,\ldots,$
and a given $x_0$. \bigskip

We denote the optimal sequence of pairs of state variables and controls for problem \eqref{eq:state}-\eqref{eq:objMax} by $\{x^*_{t+1},u^*_t\},~t=0,1,2,\ldots$, i.e. $\{u^*_t\} \in \textrm{FC}(x_0)$, and $$v(x_0)=J(x_0,\mathbf{u^*}),~\mathbf{u^*}=\{u^*_t\}.$$
\end{frame}



\begin{frame}[fragile]
\frametitle{Necessary conditions for optimality}
A variation of the familiar approach for problems of this type is the following:
\begin{block}{Algorithm}
\begin{enumerate}
\item Construct the Lagrangian
\[\mathcal{L}(x_1,x_2,\ldots,u_0,u_1,\ldots) =
 \sum_{t=0}^{\infty}\beta^t \left[ F(x_t,u_t) + \lambda'_t 
[f(x_t,u_t)-x_{t+1}] \right],  \] where $\lambda_t=(\lambda^1_t,\ldots,\lambda^n_t)'$, $t=0,1,2,\ldots,$ are the Lagrange multipliers and the prime ($ ' $) denotes transposition as usual: $$\lambda'_t  [f(x_t,u_t)-x_{t+1}]
= \sum_{i=1}^n \lambda_t^i [f^i(x_t,u_t)-x^i_{t+1}].$$
\end{enumerate}
\end{block}
\end{frame}



\begin{frame}[fragile]
\frametitle{Necessary conditions for optimality}
\begin{block}{Algorithm (cont.)}
\begin{enumerate} \setcounter{enumi}{1}
\item Differentiate $\mathcal{L}$ w.r.t. $x_t$ and $u_t$, set the resulting expressions equal to zero and obtain first-order necessary conditions for optimality: \begin{equation}\begin{split}
& \beta \left[ F_{x_t^k}(x_t,u_t)+\sum_{i=1}^n \lambda_t^i f^i_{x_t^k}(x_t,u_t)\right] = \lambda^k_{t-1},~ k=1,\ldots,n, \\
& F_{u_t^j}(x_t,u_t)+ \sum_{i=1}^n \lambda_t^i f^i_{u_t^j}(x_t,u_t) = 0,~ j=1,\ldots,m.
\end{split}
\label{eq:FOCs}
\end{equation}
\item Equations \eqref{eq:state} and \eqref{eq:FOCs} are solved as a system and we obtain a candidate solution $\{u_t\}_{t=0}^\infty$ or, more precisely, a sequence
$\{x_{t+1},u_t\}_{t=0}^\infty$. 
\end{enumerate}
\end{block}\bigskip

\textbf{Note:} It is common to find a stationary point of the system \eqref{eq:state} and \eqref{eq:FOCs}, and work with a linearised version of the system around that point.
\end{frame}


\begin{frame}[fragile]
\frametitle{Necessary conditions for optimality}
In matrix notation the above takes the form:
\[ \mathcal{L}_x = \beta^t \nabla_x F(x_t,u_t)+\beta^t \nabla'_x f(x_t,u_t) \lambda_t - \beta^{t-1}\lambda_{t-1} = 0 \Rightarrow \]
\begin{equation}
\beta ( \nabla_x F(x_t,u_t)+ \nabla'_x f(x_t,u_t) \lambda_t ) = \lambda_{t-1}.
\label{eq:LagrX}
\end{equation}  \bigskip
\[ \mathcal{L}_u = \beta^t \nabla_u F (x_t,u_t)+ \beta^t \nabla'_u f(x_t,u_t) \lambda_t = 0 \Rightarrow \]
\begin{equation}
\nabla_u F (x_t,u_t)+ \nabla'_u f(x_t,u_t) \lambda_t = 0
\label{eq:LagrU}
\end{equation}
\end{frame}



\begin{frame}[fragile]
\frametitle{Necessary conditions for optimality}
\textbf{Why is this algorithm valid?}\bigskip \bigskip

The value function for problem \eqref{eq:state}-\eqref{eq:objMax} satisfies a version of the Bellman equation:
\begin{equation}
v(x) = \max_{u \in \Omega(x)}  \left[
F(x,u)+\beta v(f(x,u)) \right]. \label{eq:Bellman}
\end{equation}


Let the maximum in \eqref{eq:Bellman} be attained on the interior of the set $ F(x) $. Denote this point by $ u = \nu(x) $ and assume that all objects used below are differentiable.
\end{frame}



\begin{frame}[fragile]
\frametitle{Necessary conditions for optimality}
We have \begin{equation}
 v(x) = F(x,\nu(x)) + \beta v(f(x,\nu(x))) .
\label{eq:Bellman1}
\end{equation}

Also the extremum condition is \begin{equation}
\nabla_u F(x,\nu(x)) + \beta \nabla'_u f(x,\nu(x))  \nabla v(f(x,\nu(x))) = 0.
\label{eq:diffCntrl}
\end{equation} 

Differentiating \eqref{eq:Bellman1} w.r.t. $ x $, we obtain
\begin{equation*}
\begin{split}
 \nabla v(x) = & \nabla_x F(x,\nu(x)) +  \nabla' \nu(x)  \nabla_u F(x,\nu(x)) + \\
 & \beta \left[ \nabla'_x f(x,\nu(x)) + \nabla' \nu(x)  \nabla'_u f(x,\nu(x))  \right] \nabla v(f(x,\nu(x)))\\
 = & \nabla_x F(x,\nu(x)) + \beta  \nabla'_x f(x,\nu(x)) \nabla v(f(x,\nu(x))) + \\
 &  \underbrace{\nabla' \nu(x)  \nabla_u F(x,\nu(x)) + \beta \nabla' \nu(x)  \nabla'_u f(x,\nu(x)) \nabla v(f(x,\nu(x)))}_{=0 \text{ in view of \eqref{eq:diffCntrl}}}.
\end{split}
\end{equation*}
\end{frame}


\begin{frame}[fragile]
\frametitle{Necessary conditions for optimality}
We thus get \begin{equation}
\nabla v(x) = \nabla_x F(x,\nu(x)) + \beta  \nabla'_x f(x,\nu(x)) \nabla v(f(x,\nu(x))).
\label{eq:Bellman1dx1}
\end{equation}
For $ x=x^*_t $ and $ u^*_t = \nu(x^*_t) $, equations \eqref{eq:diffCntrl} and \eqref{eq:Bellman1dx1} take the form
\begin{equation}
\nabla_u F(x^*_t,u^*_t) + \beta \nabla'_u f(x^*_t,u^*_t)  \nabla v(x^*_{t+1}) = 0,
\label{eq:diffCntrlA} 
\end{equation}
\begin{equation}
\nabla v(x^*_t) = \nabla_x F(x^*_t,u^*_t) + \beta  \nabla'_x f(x^*_t,u^*_t) \nabla v(x^*_{t+1}).
\label{eq:Bellman1dx1A}
\end{equation}

Set $ \lambda_t := \beta \nabla v(x^*_{t+1}) $ in \eqref{eq:diffCntrlA} and \eqref{eq:Bellman1dx1A}, to obtain precisely \eqref{eq:LagrX} and \eqref{eq:LagrU}.
\end{frame}


\begin{frame}[fragile]
\frametitle{Example: NCs for a problem with explicit controls}
Compute the NCs for the problem:
\[ \max_{\{c_t\}} \sum_{t=0}^{\infty}\beta^t \ln c_t  \]
\[ k_{t+1} = (1-\delta)k_t + y_t - c_t,\quad k_0>0 \text{ -- given} \] \[ y_t=A k_t^\alpha , \quad 0<\underline{\epsilon} \leq c_t \leq \overline{\epsilon}, \quad \alpha,\beta \in (0,1), \quad A>0\]\bigskip \pause

The Lagrangian for the problem is \[ \mathcal{L} = \sum_{t=0}^{\infty}\beta^t \left[\ln c_t + \lambda_t ((1-\delta)k_t + Ak_t^\alpha - c_t -k_{t+1}) \right]. \]

\[ \dfrac{\partial \mathcal{L}}{\partial c_t} = \beta^t \dfrac{1}{c_t}- \beta^t \lambda_t = 0 \quad \Rightarrow \quad \lambda_t = \dfrac{1}{c_t}. \]
\end{frame}

\begin{frame}[fragile]
\frametitle{Example: NCs for a problem with explicit controls}
\[\begin{split}
\dfrac{\partial \mathcal{L}}{\partial k_t} =& \beta^t \lambda_t (1-\delta + \alpha A k_t^{\alpha-1}) - \beta^{t-1}\lambda_{t-1} = 0 \quad \Rightarrow \quad \\
&\lambda_{t-1} = \beta \lambda_t (1-\delta + \alpha A k_t^{\alpha-1}).
\end{split} \]

Substituting the $ 1/c_t $ for $ \lambda_t $, we obtain \[ c_t = \beta c_{t-1} (1-\delta + \alpha A k_t^{\alpha-1}). \]

This is a version of a \emph{consumption Euler equation} showing how consumption changes between two periods.%\bigskip \pause

%\textbf{Homework:} Use the state equation for the model to eliminate $ c_t $ from the above consumption Euler equation and obtain a difference equation involving only $ k $. How does this result differ from the result for the same example in variational form?
\end{frame}


\begin{frame}[fragile]
\frametitle{Sufficient conditions}
\begin{Fact} Let $\{\lambda_t\}$ and $\{x^*_{t+1},u^*_t\}$,
$t=0,1,2,\ldots,$ be obtained by using \eqref{eq:state}
and \eqref{eq:FOCs}. If 
	\begin{enumerate}
	\item The functions $F(x,u)$ and $f(x,u)$
	are concave in $(x,u)$,
	\item The Lagrange multipliers $\lambda_t^1,\ldots,\lambda_t^n,
	~t=0,1,2,\ldots$ are nonnegative,
	\item The state space $X$ is a subset of $\mathbb{R}^n_+$ and the following transversality condition is valid $$\lim_{T\rightarrow \infty} \beta^T \lambda'_T 
	x^*_{T+1}=0,$$
	\end{enumerate} 
then the sequence $\{x^*_{t+1},u^*_t\}$ (for a given $x_0$) is optimal for problem  \eqref{eq:state}-\eqref{eq:objMax}.
\label{fc:sufficiency}\end{Fact}
\end{frame}


\begin{frame}[fragile]
\frametitle{Sufficient conditions}
We shall verify the validity of Fact \ref{fc:sufficiency}.\bigskip

Recall that \eqref{eq:FOCs} in matrix terms is given by \eqref{eq:LagrX} and \eqref{eq:LagrU}.

Consider \[ \mathcal{L}_T(x_t,u_t) = \sum_{t=0}^{T}\beta^t \left\{ F(x_t,u_t) + \lambda'_t  [f(x_t,u_t)-x_{t+1}] \right\} .\]

We have \begin{equation}
\begin{split}
& D := \mathcal{L}_T (x_t,u_t) - \mathcal{L}_T(x^*_t,u^*_t) = \sum_{t=0}^{T}\beta^t \lambda'_t  (x^*_{t+1} - x_{t+1} )+  \\
& \sum_{t=0}^{T}\beta^t [ F(x_t,u_t) + \lambda'_t  f(x_t,u_t) 
- F(x^*_t,u^*_t) - \lambda'_t  f(x^*_t,u^*_t) ].
\end{split}
\label{eq:LagrDiff}
\end{equation}
\end{frame}


\begin{frame}[fragile]
\frametitle{Sufficient conditions}
Then, in view of concavity, we get \begin{equation*}
\begin{split}
&\mathcal{L}_T (x_t,u_t) - \mathcal{L}_T(x^*_t,u^*_t) ~~(= D)~~ \leq \sum_{t=0}^{T}\beta^t \lambda'_t  (x^*_{t+1} - x_{t+1} )+  \\
&\sum_{t=0}^{T}\beta^t [ \nabla'_x F(x^*_t,u^*_t)(x_t-x_t^*) + \nabla'_u F(x^*_t,u^*_t) (u_t-u_t^*) +\\
& \lambda'_t  [\nabla_x f(x^*_t,u^*_t)(x_t-x_t^*) + \nabla_u f(x^*_t,u^*_t) (u_t-u_t^*) ]] = \\
& \sum_{t=0}^{T}\beta^t \lambda'_t  (x^*_{t+1} - x_{t+1} )+  \sum_{t=0}^{T}\beta^t \underbrace{[\nabla'_x F(x^*_t,u^*_t) + \lambda'_t  \nabla_x f(x^*_t,u^*_t)]}_{=\frac{\lambda'_{t-1}}{\beta} \text{ in view of }  \eqref{eq:LagrX} } (x_t-x_t^*)\\
& + \sum_{t=0}^{T}\beta^t \underbrace{[\nabla'_u F(x^*_t,u^*_t) + \lambda'_t  \nabla_u f(x^*_t,u^*_t)]}_{=\mathbf{0}' \text{ in view of } \eqref{eq:LagrU}} (u_t-u_t^*).
\end{split}
\end{equation*}
\end{frame}



\begin{frame}[fragile]
\frametitle{Sufficient conditions}
We therefore have \begin{equation*}
\begin{split}
 D \leq &   \sum_{t=0}^{T}\beta^t \lambda'_t  (x^*_{t+1} - x_{t+1} ) +  \sum_{t=0}^{T}\beta^t \frac{\lambda'_{t-1}}{\beta}\underbrace{(x_t-x_t^*)}_{\text{N.B.: } x_0 = x^*_0} = \\
&  \sum_{t=0}^{T}\beta^t \lambda'_t  (x^*_{t+1} - x_{t+1} ) +  \sum_{t = {\color{red}1}}^{T}\beta^{t-1} \frac{\lambda'_{t-1}}{\beta}(x_t-x_t^*)= \\
&  \sum_{t=0}^{T}\beta^t \lambda'_t  (x^*_{t+1} - x_{t+1} ) +  \sum_{t = {\color{red} 0}}^{{\color{red} T-1}}\beta^{{\color{red} t}} \frac{\lambda'_{{\color{red} t}}}{\beta}(x_{{\color{red} t+1}}-x_{{\color{red} t+1}}^*)= \\
& \underbrace{\sum_{t=0}^{{\color{red} T-1}}\beta^t \lambda'_t  (x^*_{t+1} - x_{t+1}) +  \sum_{t = 0}^{T-1}\beta^{t} \frac{\lambda'_{t}}{\beta}(x_{ t+1}-x_{t+1}^*)}_{=0} + \\
& \beta^T \lambda'_T  (x^*_{T+1} - x_{T+1})  \underbrace{\leq}_{\text{since }\lambda_t,x_t \geq \mathbf{0}} \beta^T \lambda'_T  x^*_{T+1}.
\end{split}
\end{equation*}
\end{frame}


\begin{frame}[fragile]
\frametitle{Sufficient conditions}
In view of the transversality condition, we have: \begin{equation*}
 D \leq  \beta^T \lambda'_T  x^*_{T+1} ~ \mathop{\longrightarrow}_{T \rightarrow \infty} ~0 ,
\end{equation*}
i.e. 
\[  \mathcal{L}_T(x^*_t,u^*_t) - \mathcal{L}_T (x_t,u_t) \geq 0, \] which proves the optimality of the sequence
$\{x^*_{t+1},u^*_t\}$.
\end{frame}
\end{section}

%\begin{frame}[fragile]
%\frametitle{Homework}
%\framesubtitle{}
%Derive the NCs for the problem 
%\[ \sum_{t=0}^{\infty}\beta^t \ln (x_t-u_t) \]
%s.t.
%\[ x_{t+1}=u_t,\qquad x_0>0,\quad u_t>0. \]
%
%Express the result as a system of difference equations for $ x $ and $ \lambda $.
%\end{frame}


\begin{frame}[fragile]
\frametitle{Readings}
Additional readings:\bigskip

Stokey, Lucas and Prescott. 1989. \emph{Recursive methods in economic dynamics}. Chapters 2 and 4.\bigskip

Syds\ae{}ter et al. [SHSS] \emph{Further mathematics for economic analysis}. Chapter 12.\bigskip
\end{frame}

\end{document}

