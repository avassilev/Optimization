% !TeX spellcheck = en_GB
\documentclass[10pt]{beamer}
\usetheme{CambridgeUS}
%\usetheme{Boadilla}
\definecolor{myred}{RGB}{163,0,0}
%\usecolortheme[named=blue]{structure}
\usecolortheme{dove}
\usefonttheme[]{professionalfonts}
\usepackage[english]{babel}
\usepackage{amsmath,amsfonts,amssymb}
\usepackage{xcolor}
\usepackage{bm}
\usepackage{gensymb}
\usepackage{verbatim} 
\usepackage{paratype}
\usepackage{mathpazo}
\usepackage{listings}
\lstset{language=Python}

\usepackage{tikz}
\usetikzlibrary{matrix}

\DeclareMathOperator*{\interior}{int}

% Number theorem environments
\setbeamertemplate{theorem}[ams style]
\setbeamertemplate{theorems}[numbered]

% Reset theorem-like environments so that each is numbered separately
\usepackage{etoolbox}
\undef{\definition}
\theoremstyle{definition}
\newtheorem{definition}{\translate{Definition}}
\newtheorem{Fact}{\translate{Fact}}

% Change colours for theorem-like environments
\definecolor{mygreen1}{RGB}{0,96,0}
\definecolor{mygreen2}{RGB}{229,239,229}
\setbeamercolor{block title}{fg=white,bg=mygreen1}
\setbeamercolor{block body}{fg=black,bg=mygreen2}



\alt<presentation>
{\lstset{%
  basicstyle=\footnotesize\ttfamily,
  commentstyle=\slshape\color{green!50!black},
  frame = single,  
  keywordstyle=\bfseries\color{blue!50!black},
  identifierstyle=\color{blue},
  stringstyle=\color{orange},
  %escapechar=\#,
  showstringspaces = false,
  showtabs = false,
  tabsize = 2,
  emphstyle=\color{red}}
}
{
  \lstset{%
    basicstyle=\ttfamily,
    keywordstyle=\bfseries,
    commentstyle=\itshape,
    escapechar=\#,
    showtabs = false,
	tabsize = 2,
    emphstyle=\bfseries\color{red}
  }
} 

\title{R401: Statistical and Mathematical Foundations \bigskip}
\subtitle{\textcolor{myred}{Infinite-Horizon Stochastic Optimal Control in Discrete Time}}
\author{Andrey Vassilev}

\date{} 
    
\AtBeginSection{\frame{\usebeamerfont{section title}\centering\insertsection}}

\begin{document}
\maketitle



\begin{frame}[fragile]
\frametitle{Lecture Contents}
\tableofcontents
\end{frame}

\begin{section}{Introduction}\label{sec:intro}

\begin{frame}
\frametitle{Introduction}
\begin{itemize}\itemsep1em
\item Economic decision-making is intrinsically done in an uncertain environment.
\item A realistic modelling framework should therefore take uncertainty on board in some manner.
\item A standard approach to modelling decision-making under uncertainty is to use dynamic optimization problems with a stochastic element.
\item The popular dynamic stochastic general equilibrium (DSGE) models are a prominent example of this approach.
\item The presence of randomness creates differences in comparison with deterministic problems. In particular, the concept of an optimal sequence of controls has to be replaced by the concept of an optimal decision rule (policy).
\end{itemize}
\end{frame}


\begin{frame}
\frametitle{Introduction}
\begin{itemize}\itemsep1em
\item In constructing and solving a dynamic stochastic model or, specifically, a DSGE model, one typically goes through several stages:
\begin{itemize}\itemsep1em
  \item model formulation -- goal $\rightarrow$ mechanisms, shocks (structural interpretation)
  \item deriving NCs for optimality
  \item computing a steady state -- deterministic steady states 
  \item linear approximations of the model around the steady state -- log-linearization 
  \item calibration/estimation 
  \item forecasts, simulations
\end{itemize}
\end{itemize}
\end{frame}



\begin{frame}
\frametitle{Introducing randomness in an optimization problem with explicit controls}
Suppose that the state equations also include a random variable or vector:
\begin{equation}
x_{t+1}=f(x_t,u_t,{\color{red}\epsilon_t}),\quad x_0 \text{ -- given} \label{eq:statestoch}
\end{equation}

Then the state variables $ \{x_t\} $ also become random, therefore $ F(x_t,u_t) $ will be random.

This requires a modification of the objective functional. We need to take its expectation (or, more precisely, conditional expectation given information available at the respective moment). Assume that that we are at $ t=0 $:
\begin{equation}
\max_{\{u_t\}_{t=0}^\infty}{\color{red}E_0}\sum_{t=0}^\infty
\beta^t F(x_t,u_t)\label{eq:objstoch}
\end{equation}
\end{frame}



\begin{frame}
\frametitle{Introducing randomness in an optimization problem with explicit controls }
\begin{itemize}
\item Sequencing of events within a given time period:\[ x_t \quad \rightarrow \quad u_t \quad \rightarrow \quad \epsilon_t \quad \Longrightarrow \quad x_{t+1}\quad \rightarrow \quad \cdots \] This means that, when making the decision, we are not sure where we'll be next period.

Some sources write $ x_{t+1}=f(x_t,u_t,\epsilon_{t+1}) $ to make this sequencing more explicit.
\item Unlike deterministic models, the solution here cannot be a predetermined path $ u^*_0, u^*_1, \ldots $ For instance, it is impossible to determine the optimal $ u_5 $ when we are at time 0, since we do not know the value of $ x_5 $ (recall that it is now random).
\item For this reason the optimal mode of behaviour is given as a \emph{reaction function (rule, policy function)} $ u_t=\mu(x_t) $, which depends on the state $ x_t $ of the system. 
\end{itemize}
\end{frame}


\begin{frame}
\frametitle{Introducing randomness in an optimization problem with explicit controls }
\begin{itemize}\itemsep1em
\item Under this structure of the model, decisions are taken only for the current moment.
\item Everything else is a plan for the future which may or may not be carried out as envisaged.
\item As time advances, the model is re-solved over and over for the new conditions that materialize as a result of the combination of past decisions and random disturbances.
\item This is reflected in the way the conditional mathematical expectation is applied when computing NCs for optimality.
\end{itemize}
\end{frame}
\end{section}

\begin{section}{Necessary conditions}\label{sec:NCs}

\begin{frame}
\frametitle{Necessary conditions for optimality}\framesubtitle{General formulation
(algorithmic solution)} We have the following problem
$$\max_{u_{s+t}}E_s\sum_{t=0}^\infty \beta^t F(x_{s+t},u_{s+t})$$ s.t.
$$x_{s+t+1}=f_{s+t}(x_{s+t},u_{s+t},\epsilon_{s+t}),\quad x_s\textrm{ --
given},$$ where $x_{s+t}\in \mathbb{R}^n$ and $u_{s+t}\in \mathbb{R}^m$.

As usual, we construct the Lagrangian $$\mathcal{L}=E_s\sum_{t=0}^\infty \beta^t
\left\{ F(x_{s+t},u_{s+t}) + \lambda'_{s+t}
(f_{s+t}(x_{s+t},u_{s+t},\epsilon_{s+t})-x_{s+t+1}) \right\}$$

\end{frame}


\begin{frame} \frametitle{Necessary conditions for optimality}\framesubtitle{General formulation
(algorithmic solution)} Under certain conditions, which are fulfilled for typical economic problems, the solution will satisfy 
$$\frac{\partial \mathcal{L}}{\partial u_{s+t}^i}=0,~i=1,\ldots,m; \quad \frac{\partial \mathcal{L}}{\partial x_{s+t+1}^j}=0,~j=1,\ldots,n,~\forall t \geq 0,\textrm{ i.e.} $$
$$E_s
\left\{ \frac{\partial}{\partial u_{s+t}^i}F(x_{s+t},u_{s+t}) + \sum_{k=1}^n \lambda^k_{s+t}
\frac{\partial}{\partial u_{s+t}^i}f^k_{s+t}(x_{s+t},u_{s+t},\epsilon_{s+t}) \right\}=0,$$
\begin{equation*}
    \begin{split}
\beta E_s
\Biggl\{ & \frac{\partial}{\partial x_{s+t+1}^j} F(x_{s+t+1},u_{s+t+1}) + \\ &  \sum_{k=1}^n \lambda^k_{s+t+1}
\frac{\partial}{\partial x_{s+t+1}^j}f^k_{s+t}(x_{s+t+1},u_{s+t+1},\epsilon_{s+t+1}) \Biggr\} =  E_s
\left\{ \lambda_{s+t}^j \right\}
    \end{split}
\end{equation*}
\end{frame}



\begin{frame}
\frametitle{Necessary conditions for optimality}
\framesubtitle{Practical advice and comments}
\begin{itemize}\itemsep1em
\item It is easier to write the problem for the case $ s=0 $. Standard problems are invariant with respect to time and we lose no generality. 
\item It is convenient to apply the expectation operator at the end of the computation, i.e. after differentiation and the subsequent transformations.
\item We differentiate w.r.t. all controls at an arbitrary moment $ t $ and w.r.t. to the state variables at $ t+1 $, then set the results equal to zero.
\item When we apply the conditional expectation operator, it is applied at time $ t $.
\end{itemize}
\end{frame}



\begin{frame}
\frametitle{Necessary conditions for optimality}
\framesubtitle{Practical advice and comments}
\begin{itemize}\itemsep1em
\item In addition to the above conditions, a solution to the model also has to satisfy a stochastic version of the transversality condition. We ignore this here but bear in mind that it is another requirement that has to be checked for a given model.
\item Sometimes the state equations are written in the form \[ x_t = f_t(x_{t-1},u_t,\epsilon_t). \] (This can be interpreted as measuring the state variables at the end of each period rather than the beginning of the next one, i.e. 31 December instead of 1 January.) In such cases we have to differentiate w.r.t. the state variables at time $ t $ instead of $ t+1 $.
\end{itemize}
\end{frame}

\end{section}


\begin{section}{Examples}\label{sec:ex}

\begin{frame}
\frametitle{Example I}
\framesubtitle{}
Problem:
\[ \max_{\{C_t\}} E_0 \sum_{t=0}^{\infty}\beta^t \frac{C_t^{1-\sigma}}{1-\sigma} \]
\[ A_{t+1}=(1+r)A_t + Y_t - C_t,~Y_t=\bar{Y}-\epsilon_t \]
$ C_t $ -- consumption, $ A_t $ -- assets, $ Y_t $ -- income, $ r $ -- interest rate \bigskip

Lagrangian:\[ \mathcal{L} = \sum_{t=0}^{\infty}\beta^t \left\{\frac{C_t^{1-\sigma}}{1-\sigma}+\lambda_t \left[(1+r)A_t + Y_t - C_t - A_{t+1}\right] \right\} \]
\end{frame}




\begin{frame}
\frametitle{Example I}
\framesubtitle{}
Necessary conditions:
\[ \frac{\partial \mathcal{L} }{\partial C_t} = \beta^t \{C_t^{-\sigma}-\lambda_t\} = 0 \quad\Longrightarrow\quad C_t^{-\sigma}=\lambda_t \]
\[ \frac{\partial \mathcal{L} }{\partial A_{t+1}} = -\beta^t \lambda_t + \beta^{t+1}\lambda_{t+1}(1+r) = 0 \quad\Longrightarrow\quad \lambda_t = \beta(1+r)\lambda_{t+1} \]

We apply the expectation operator $ E_t $:
\[ E_t \{C_t^{-\sigma}\}=E_t \{\lambda_t\} \quad\Longrightarrow\quad C_t^{-\sigma}=\lambda_t \]
\[ E_t \{\lambda_t\} = E_t\{\beta(1+r)\lambda_{t+1}\} \quad\Longrightarrow\quad \lambda_t = \beta(1+r)E_t\{\lambda_{t+1}\} \]

Substituting $ \lambda_t = C_t^{-\sigma} $ in the last expression:
\[ C_t^{-\sigma} = \beta (1+r)E_t \left\{C_{t+1}^{-\sigma}\right\} \] %\pause
%\textbf{Homework:} Solve Example I for the case when the state equation is presented according to the convention $A_{t}=(1+r)A_{t-1} + Y_t - C_t$.
\end{frame}


\begin{frame}
\frametitle{Example II}
\framesubtitle{}
Problem:
\[ \max_{\{C_t\},\{N_t\}} E_0 \sum_{t=0}^{\infty}\beta^t U(C_t,N_t) \]
\[ Q_t B_t = B_{t-1} + W_t N_t - P_t C_t - T_t  \]
$ C_t $ -- consumption, $ N_t $ -- labour (hours worked), $ B_t $ -- discount bonds purchased at time $ t $ and maturing at $ t+1 $, $ Q_t $ -- bond price, $ P_t $ -- price of consumer good, $ W_t $ -- wages, $ T_t $ -- taxes or transfers (depending on sign) \bigskip

Lagrangian:\[ \mathcal{L} = \sum_{t=0}^{\infty}\beta^t \left\{U(C_t,N_t)+\lambda_t \left[ B_{t-1} + W_t N_t - P_t C_t - T_t - Q_t B_t \right] \right\} \]
\end{frame}




\begin{frame}
\frametitle{Example II}
\framesubtitle{}
Define $ U_{c,t} := \frac{\partial}{\partial C}U(C_t,N_t)$, $ U_{n,t} := \frac{\partial}{\partial N}U(C_t,N_t)$.\bigskip

Differentiate w.r.t. the controls:
\[ \frac{\partial \mathcal{L} }{\partial C_t} = \beta^t U_{c,t} - \beta^t P_t \lambda_t = 0 \quad\Longrightarrow\quad \lambda_t = \frac{U_{c,t}}{P_t} \]
\[  \frac{\partial \mathcal{L} }{\partial N_t} = \beta^t U_{n,t}+\beta^t \lambda_t W_t =0 \quad\Longrightarrow\quad \lambda_t W_t = -U_{n,t}  \]\bigskip

Combining the previous two equations, we obtain \[ \frac{W_t}{P_t} = -\frac{U_{n,t}}{U_{c,t}}. \]
\end{frame}




\begin{frame}
\frametitle{Example II}
\framesubtitle{}
Differentiate w.r.t. the state variable (at time $ t $ in this case!):
\[ \frac{\partial \mathcal{L} }{\partial B_t} = -\beta^t Q_t \lambda_t + \beta^{t+1}\lambda_{t+1} = 0 \quad\Longrightarrow\quad Q_t \lambda_t = \beta E_t \{\lambda_{t+1}\} \]

Substitute the expression for $ \lambda $ for times $ t $ and $ t+1 $ to obtain 
\[ Q_t = \beta E_t \left\{ \frac{U_{c,t+1}}{U_{c,t}}\frac{P_t}{P_{t+1}} \right\} .  \]

We ultimately get \[ \frac{W_t}{P_t} = -\frac{U_{n,t}}{U_{c,t}}, \quad Q_t = \beta E_t \left\{ \frac{U_{c,t+1}}{U_{c,t}}\frac{P_t}{P_{t+1}} \right\} \text{ for } t=0,1,2,\ldots \]

%\textbf{Homework:} What do the above NCs look like for the case when the utility function has the form $ U(C_t,N_t)=\frac{C_t^{1-\sigma}}{1-\sigma}-\frac{N_t^{1+\varphi}}{1+\varphi},~\sigma>0,~\varphi \geq 1 $?
\end{frame}

\end{section}


\begin{section}{Linear approximation and log-linearization}\label{sec:llin}

\begin{frame}
\frametitle{Linear approximations}
\framesubtitle{General remarks}
\begin{itemize}\itemsep1em
\item When solving a dynamic economic model it is often convenient to work with an approximation to the true equations of the model.
\item We are usually interested in the properties of the model in a neighbourhood of an equilibrium point (steady state), therefore the approximation is taken around such a point. 
\item These approximations are typically linear: we transform the original, nonlinear equation in a linear one using some technique.
\end{itemize}
\end{frame}


\begin{frame}
\frametitle{Linear approximations}
\framesubtitle{Ways to construct a linear approximation}
\begin{itemize}\itemsep1em
\item Taking logs (for positive variables):
\[ \frac{W_t}{P_t} = C_t^\sigma N_t^\varphi \quad\Longrightarrow\quad w_t - p_t = \sigma c_t + \varphi n_t, \] where $ w_t := \ln W_t $, $ p_t := \ln P_t $ etc. 
\item Direct linearization using Taylor's formula (we'll work it out as needed)
\item Log-linearization (we'll study that next)
\end{itemize}
\end{frame}




\begin{frame} \frametitle{Log-linearization}
Let $x_t \neq 0$ and let $\bar{x}$ be one of the possible values $x$ can take (it is usually a steady state value).

The \emph{logarithmic deviation (log-deviation)} of $x_t$ from $\bar{x}$ is defined as: $$\hat{x}_t := \ln \left(\frac{x_t}{\bar{x}}\right).$$

In view of $$\ln (1+z)\approx \ln (1+z_0)+ \frac{1}{1+z_0}(z-z_0),$$ taking $z_0=0$ we obtain $\ln (1+z) \approx z$ and therefore $$\ln \left(\frac{x_t}{\bar{x}}\right) = \ln \left(1+\frac{x_t-\bar{x}}{\bar{x}}\right)\approx \frac{x_t-\bar{x}}{\bar{x}}. $$
\end{frame}

\begin{frame} \frametitle{Log-linearization}
Let $y_t=f(x_t,z_t)$, where $f$ is differentiable. \bigskip

Then \begin{equation}\label{eq:mainLLin}
\hat{y}_t = \frac{f_x(\bar{x},\bar{z})\bar{x}}{f(\bar{x},\bar{z})}\hat{x}_t + \frac{f_z(\bar{x},\bar{z})\bar{z}}{f(\bar{x},\bar{z})}\hat{z}_t .
\end{equation}

We have:

\begin{enumerate}\itemsep1em
  \item $y_t=x_t z_t \quad \Rightarrow \quad \hat{y}_t = \hat{x}_t + \hat{z}_t ,$
  \item $y_t=x_t / z_t \quad \Rightarrow \quad \hat{y}_t = \hat{x}_t - \hat{z}_t ,$
  \item $y_t=x_t + z_t \quad \Rightarrow \quad \hat{y}_t = \frac{\bar{x}}{\bar{y}}\hat{x}_t + \frac{\bar{z}}{\bar{y}}\hat{z}_t ,$
  \item $g(x_t,y_t)=0 \quad \Rightarrow \quad \hat{y}_t = - \frac{g_x(\bar{x},\bar{y})\bar{x}}{g_y(\bar{x},\bar{y})\bar{y}}\hat{x}_t .$
\end{enumerate}
\end{frame}


\begin{frame}[fragile]
\frametitle{Log-linearization: an example}
Log-linearize the equation \[ A_{t+1} = (1+r_t)A_t + Y_t - P_t C_t. \] \bigskip \pause

We have \[ \hat{A}_{t+1} = \dfrac{(1+r)A}{A}\widehat{(1+r_t)A_t} + \dfrac{Y}{A}\hat{Y}_t - \dfrac{PC}{A}\widehat{P_t C_t} \Rightarrow \]
\[ \hat{A}_{t+1} = (1+r)(\widehat{(1+r_t)}+\hat{A}_t) + \dfrac{Y}{A}\hat{Y}_t - \dfrac{PC}{A}(\hat{P}_t + \hat{C}_t)  \Rightarrow \]
\[ \hat{A}_{t+1} = (1+r)\left (\dfrac{r}{1+r}\hat{r}_t+\hat{A}_t\right ) + \dfrac{Y}{A}\hat{Y}_t - \dfrac{PC}{A}(\hat{P}_t + \hat{C}_t)  \Rightarrow \]
\[ \hat{A}_{t+1} = r\hat{r}_t + (1+r)\hat{A}_t + \dfrac{Y}{A}\hat{Y}_t - \dfrac{PC}{A}(\hat{P}_t + \hat{C}_t)  . \]
\end{frame}


\end{section}



\begin{frame}[fragile]
\frametitle{Readings}
Additional readings:\bigskip

Syds\ae{}ter et al. [SHSS] \emph{Further mathematics for economic analysis}. Chapter 12.\bigskip

John Stachurski. \emph{Economic Dynamics --  Theory and Computation}. MIT Press, 2009 (Not easy but good mix between theory and computational applications. Contains an introduction to Python + code.)\bigskip

\textit{For the brave:}

Stokey and Lucas. \emph{Recursive methods in economic dynamics}. Part III.\bigskip

D. Bertsekas, S. Shreve. \emph{Stochastic Optimal Control: The Discrete-Time Case}. New York, Academic Press, 1978.\bigskip
\end{frame}

\end{document}

